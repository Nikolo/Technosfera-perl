\documentclass{report}
\usepackage{expl3,xargs,comment,luacode}
\usepackage[table]{xcolor}
\usepackage{polyglossia,fancyvrb}
\usepackage{enumitem,graphicx,float} %subcaption
\usepackage{amsmath,amssymb,mathtools}
\usepackage{geometry}
\geometry{margin=2cm}


\usepackage{titlesec}
\usepackage{titleps}
\definecolor{gray75}{gray}{0.75}
\newcommand{\hsp}{\hspace{20pt}}


\setdefaultlanguage{russian}
\setotherlanguage{english}
\setmainfont{CMU Serif}
\setsansfont{CMU Sans Serif}
\setmonofont{CMU Typewriter Text}


\usepackage[unicode]{hyperref}
\definecolor{linkcolor}{HTML}{000000} 
\definecolor{urlcolor}{HTML}{333377}
\hypersetup{pdfstartview=FitH, linkcolor=linkcolor, urlcolor=urlcolor, colorlinks=true}

\usepackage{tikz}
\usetikzlibrary{fadings,scopes,chains,decorations,decorations.pathreplacing,angles,calc,quotes,positioning}
\usepackage{pgfplots,unicode-math,wasysym}

\usepackage{shellesc}

\usepackage[outputdir=build]{minted}
\setminted{breaklines=true}

\addto\captionsrussian{%
  \renewcommand{\chaptername}{Лекция}
}


\expandafter\def\csname ver@transparent.sty\endcsname{}
\expandafter\def\csname ver@subfig.sty\endcsname{}
\usepackage{svg}


%\makeatletter
%\expandafter\def\csname PYGdefault@tok@err\endcsname{\def\PYGdefault@bc##1{{\strut ##1}}}
%\makeatother
\AtBeginEnvironment{minted}{\renewcommand{\fcolorbox}[4][]{#4}}

\makeatletter
\input{common/tikz/common_debug.tikz}
\input{common/tikz/tikz_svhead.tikz}
\input{common/tikz/tikz_stackblock.tikz}
\makeatother
\debugfalse


\tikzset{
grammar style/.style={
  text depth=.25ex,
  line width=2pt,
  draw=black,
  rounded corners=3mm,
% Настройка типов нодов:
  common/.style  = { minimum size=6mm,  draw=black!50, font=\large\rm\bf,
                      top color=white,  bottom color=green!20},
  value/.style          = { common, draw,
                            minimum width=25mm,
                            rounded corners=0mm      },
  exotic value/.style   = { value,  font=\large\rm,
                            rounded corners=3mm      },
  operator/.style       = { common, circle           },
% Макрос, который устанавливает референтные точки автоматически
  setSyntaxDiagramPoints/.code = {
    \path (-6,0) coordinate (LEnd)      ( 6,0) coordinate (REnd)
          (-5,0) coordinate (LBracket)  ( 5,0) coordinate (RBracket)
        (-3.8,0) coordinate (L)       ( 3.8,0) coordinate (R)
        (-4.1,0) coordinate (LP)      (4.1 ,0) coordinate (RP)
        (0, 0.8) coordinate (UP)      (0, -1)  coordinate (DW);
  }
}}


\input{tikz/tikz-svschemes.tikz}

%\includeonly{lectures/L7/L7,tex/L10}
\newcommand\perldoc[1]{\href{http://perldoc.perl.org/#1.html}{#1}}

\begin{document}
\tableofcontents
%\chapter{Введение в программирование на Perl}
\section{История}
\subsection{Появление языка Perl}
18 декабря 1987г. — вышла первая версия языка программирования Perl, создан он был программистом Ларри Уоллом (Larry Wall). В названии этого языка кроется аббревиатура practical extraction and report language. Однако не сложно заметить, что в аббревиатуре не хватает одной буквы «a» (PEARL). Но затем стало известно, что такой язык существует, и букву «a» Ларри решил убрать, тем самым почти не изменив звичание. Символом языка Perl является верблюд — не слишком красивое, но очень выносливое животное, способное выполнять тяжёлую работу.

Целью Ларри Уолла никогда не было получение денег. Напротив, он внёс существенный вклад в культуру бесплатного распространения программ с их исходными кодами как средств работы программистов. Новый язык программирования Уолл разрабатывал для того, чтобы решить проблемы программистов, с которыми он сам сталкивался в течение рабочего дня. Когда первая версия языка вышла в свет, Ларри Уолл обеспечил открытый доступ и к исходному коду самой программы. Любой желающий может бесплатно скачать и пользоваться Perl независимо от того, нужен ли он ему для усовершенствования собственной странички или для создания мультимилионного Интернет-проекта. 

Перл создавался в среде Unix, которая оказала существенное влияние на развитие языка и его популярность. Среда Unix изначально создавалась группой программистов для самих же себя — удобное рабочее место программиста. Перл перенял такие принципы как:
\begin{itemize}
 \item максимально функционально
 \item кратко
 \item единообразно
\end{itemize}

\subsection{Становление языка Perl}
Благодаря языку Perl стартовал Yahoo — проект, авторам которого прекрасно удаётся заработок на сайте. С его же помощью создан Amazon и миллионы других сайтов.

Однако новому языку для обработки текстов было не просто. Как и упоминалось ранее, Perl стартовал в среде unix, а в ней уже существовали другие утилиты по обработке текста: awk, sed, grep и другие. Собственно, которые и стали толчком к появлению языка Perl. Perl включал в себя всё самое нужное из имеющихся на тот момент утилит и упрощал/ускорял работц системных администраторов, за что и полюбился огромному их числу.
Так же нельзя не отметить, что Perl был лёгок в изучении и применении, т.к. в большенстве случаев он повторял синтаксис других утилит. С его помощью действительно можно было решить большинство повседневных задач

\subsection{Развитие языка Perl}
Развитие на то время было мотивированно тем, что программисты из разных стран отправляли Ларри Уоллу предложения по модернизации языка и доработкам. Аудитория использования языка росла и её потребности тоже. 
\begin{itemize}
 \item 1988 году вышла версия 2.0
 \item1989 году вышла версия 3.0
 \item1991 году вышла версия 4.0
 \item1994 году появляется знаменитая 5 версия языка Perl
\end{itemize}
При её подготовке Ларри Уолл многое переосмыслил, и почти полностью переписал. Такие большие изменения были сделаны из-за того, что Ларри Уолл был не в состоянии принимать все доработки которые ему присылали, более того, и не успевал делать новые возможности. С этой версии Perl стал модульным в нем появились зачатки ООП
\verb|"Настоящее величие в том, сколько свободы вы даёте другим, а не в том, как вам удаётся заставлять других делать то, что вы хотите" - Ларри Уолл|
До выхода 5 версии Ларри делал Perl для сообщества и по просьбе сообщества. А начиная с версии 5.0 он стал придерживаться позиции, что теперь этот язык должно делать сообщество само для себя и под свои нужды. Что сыграло большой плюс в развитие языка для широкого применения, который следовал модным тенденциям, так и минус в том, что сам язык перестал выпускать новые версии.

До 2009 года было примерно 200 релизов 5 версии языка perl. Были даже параллельные разработки разных направлений развития языка. perl 5.005 развивался отдельно и параллельно вплоть до 2009 года. 
А вышедший в 2000 году перл версии 5.6.0 с поддержкой юникода вытеснил другие версии впитав в себя все полезное из них

На этом развитие языка не закончилось, Perl продолжает развиваться, в 2003 году появляется сайт perl6.ru

\verb|"Перл 5 уже начал умирать, потому что люди воспринимали его как тупиковый язык. Странно, но когда мы объявили Перл 6, Перл 5 неожиданно обрёл второе дыхание" - Ларри Уолл|

27 ноября 2004 года был выпущен релиз 5.8.6. В него включили все необходимое/недостающее для:
\begin{itemize}
 \item написания модулей
 \item работы с юникодом
 \item создания высокопроизводительных приложений
\end{itemize}

И с этого момента опять происходит изменение акцентов развития языка. Для большенства нужд Perl не требовал доработок вовсе. Все занялись написанием модулей, Сам Perl развивался неспешно, новые версии появлялись только, как тестовые, что бы оценить возможность внедрения новых фич или ускорение существующих, и не сломать функциональность и работоспособность большенства модулей.

Но Параллельно началась разработка perl 5.10, который впитывал в себя всё самое интересное из 6 версии.

В начале 2006 года вышла версия 5.8.8. Мотивирован выпуск этой версии был улучшениями для работы с XS модулями. В этот момент борьбы пошла не за удобство использования и универсальность, а за скорость. Так же в этот релиз вошло множество мелких исправлений, мешающих разработке новых модулей и программ.

\subsection{Perl 6}

Как уже упоминалось раннее сайт perl6.ru появился в 2003 году. Что то вменяемое и рабочее можно было написать только к 2010 году. В 2014 году стало все гораздо лучше, но в продакшене мы его еще не увидели. 25 декабря 2015 года состоялся релиз компилятора Rakudo 2015.12. И теперь можно присоединятся к немогочисленным разработчикам на perl6!

И как это не парадоксально, но perl6 сыграл очень большую роль в развитии perl5. Внедрение «современности», а именно полноценная поддержка классов была успешна реализована в perl5. Не один раз и не за один подход («Есть больше одного способа сделать это» и «Простые вещи должны оставаться простыми, а сложные — стать выполнимыми»).

\subsection{Утверждение языка}
Вернёмся немного назад язык Perl версии 5 с самого своего появления:
\begin{itemize}
 \item стал активно занимать нишу разработки WEB приложений
 \item укреплять свои позиции среди системных администраторов
 \item количество однострочников, написанных системными администраторами, всего мира не поддаётся исчислению
\end{itemize}

Однострочник: программа написанная в одной строке, как правило не более 200 символов.

В 2002 году Perl был исключён из стандартной поставки FreeBSD. Что еще в тот момент породило множество статей  о несостоятельности Perl-а и внесло смуту в рады системных администраторов. Однако обусловлено это было тем, что в стандартной поставке FreeBSD был perl версии 5.0, а в портах уже жил 5.6. Релиз инженеры FreeBSD не готовы были обновить версию Perl-а из-за того, что часть модулей не была способна работать с новой версией, а занимать актуализацией всех пакетов, а на тот момент их были сотни, было принято решение исключить вообще весь софт на Perl-е из поставки операционной системы и дать возможность системным администраторам самим выбрать нужную им версию с необходимыми модулями. Таким образом это неявно повлекло к обновлению версии Perl на множестве новых серверов. Системные администраторы выбирали более новую версию из-за того, что она была более безопастной в плане эксплуатации, а разработчики из-за того, что она была более функциональной.

До 2005 года perl занимал в нише web программирования лидирующее положения. Разработчики руководствовались тем, что PHP было мало, а Java была уже перебором выбор останавливался на Perl.

С 2005 года перл начал терять свои позиции в области веб разработки. Началом этого послужило давление со стороны PHP. Проигрыш языку, который создавался именно для web-разработки, был логичным. У языка Perl не было сравнимого по своим функциям IDE. Порог входа в PHP ниже и из-за этого стоимость разработчика на Perl была намного выше, чем на PHP. Ну и конечно же любимое в тот момент высказывание: \verb|Некрасивые ошибки вида "Internal Server Error"|

Примерно в конце 2006 года в сети Интернет стали встречаться посты «Perl умер». О Перле стали меньше говорить, но он продолжает делать свою работу. Опять таки из-за того, что не было выхода новых версий, небыло и статей, Perl на тот момент был известным языком со своим набором функций и большим кол-вом модулей, про это все уже знали и говорить об этом еще раз ни кто не хотел. Примерно в это же время появился Python 3.0, хипстеры качают новый язык, а он действительно во многом получился новым. Так же близился релиз второй версии ruby on rails. Ну и нельзя не отметить, что это высказывание было простимулированно долгим созданием 6 версии языка.

Он окунёмся в цифры:
\begin{itemize}
 \item В 2006 году было выпущено более 3000 модулей для perl
 \item В 2007 году приблизительно 5500 модулей
\end{itemize}

\begin{figure}[H] \centering
  \includegraphics[width=10cm]{lectures/L1/stats6.png}
\end{figure}\noindent

К 2008 году по всему миру было собрано много групп Perl Mongers пытаясь противостоять вытеснению языка perl. В 2014 году их было примерно 256 групп, 8 из них в России. Эти цифры уже ушли в историю, но о их нельзя было не заметить.

\begin{figure}[H] \centering
  \includegraphics[width=10cm]{lectures/L1/act-conferences.png}
\end{figure}\noindent

\verb|act| - конференции программа которых зарегистрирована и размещена на официальном сайте Moungers групп.

На конференции O'Reilly's Money по финансовым технологиям в Нью-Йорке в 2008 году подсчитали количество упоминаний докладчиками базовых технологий.

Топ 3:
\begin{itemize}
 \item Перл
 \item SQL
 \item XML
\end{itemize}

И еще немного статистических данных:

На конец 2015 года было 600 000 уникальных посетителей CPAN в месяц. По грубым оценкам в мире на 1000 человек 1 программист ~6,5 миллионов программистов. Из этого можно примерно посчитать что 1 из 10-20 пишет на Perl. И не просто пишет, а заливает свои наработки в общую библиотеку модулей.


\subsection{Рынок труда}

Публикуемые вакансии:
\begin{figure}[H] \centering
  \includegraphics[width=10cm]{lectures/L1/jobgraph.png}
\end{figure}\noindent

Отношение публикуемых вакансий к соискателям:
\begin{figure}[H] \centering
  \includegraphics[width=10cm]{lectures/L1/jobgraph1.png}
\end{figure}\noindent

Зарплаты
\begin{figure}[H] \centering
  \includegraphics[width=10cm]{lectures/L1/zp.png}
\end{figure}\noindent

\subsection{Итог}
На сегодняшний день мы имеем:
\begin{itemize}
 \item  хорошо зарекомендовавший себя язык
 \item Огромную быстро растущую библиотеку
 \item Большое, активное сообщество
 \item Язык Perl6 и его отличный прототип, который вносит свои коррективы в развитие perl5
 \item идею сделать perl7, который быстро попадёт в продакшин, основываясь на опыте создания perl6 и надобности perl5
 \item Высокие зарплаты
\end{itemize}


\section{Настройка окружения}
\subsection{Настройка окружения в Mac OS}
Сам perl уже предустановлен (узнать версию можно, если набрать \verb|perl -v| в командной строке), но для сборки сторонних XS-модулей придётся установить xcode, <<Command line tools>>. Также, возможно, придется использовать macports (это perl установленный в отдельную директорию, отличный от системного).

Подробнее про установку можно прочитать в perldoc \perldoc{perlmacosx}.

\subsection{Настройка окружения в Linux}
Как правило перл есть в репозиториях дистрибутива. Команды для установки для некоторых дистрибутивов:
\begin{itemize}
 \item для CentOS: \verb|yum install perl| (Как правило в CentOS версия perl обновляется крайне редко. Поэтому, если необходима более современная версия, ее придется собирать самостоятельно.)
 \item для Debian: \verb|apt-get install perl|
 \item для GenToo: \verb|emerge dev-lang/perl|
 \item для FreeBSD: \verb|pkg add perl|
\end{itemize}
Сборки Perl есть почти под все операционные системы, даже под некоторые микроконтроллеры. Чтобы самостоятельно собрать perl, нужно скачать исходный код и выполнить:
\begin{minted}{bash}
make perl ./Configure -des; make test instal
\end{minted}
Если нужно установить его в другую директорию, нужно указать это с помощью ключей для make.

\subsection{Настройка окружения в Windows}
Для Windows существует несколько дистрибутивов Perl:
\begin{itemize}
  \item \textbf{ActivePerl от ActiveState}. Фирма ActiveState знаменита тем, что сделала Perl продакшн-системой, модули для которой они продавали.
  \item \textbf{StrawberryPerl} (рекомендуется использовать). В отличие от ActivePerl, StrawberryPerl идёт сразу с компилятором mingw, а также существенно удобнее и проще установка модулей.
  \item \textbf{cygwin}
\end{itemize}
С ActivePerl надо позаботится о наличие nmake + win32GnuUtils иначе сборка модулей будет мучительной и утомляющей.


\subsection{Проверка доступности интепретатора}
Чтобы проверить доступность интерпретатора, необходимо набрать в терминале:
\begin{minted}{bash}
  perl -v
\end{minted}
Выдача будет примерно такой:
\begin{minted}{bash}
This is perl 5, version 18, subversion 2 (v5.18.2) built for
x86_64-linux-gnu-thread-multi
(with 40 registered patches, see perl -V for more detail)

Copyright 1987-2013, Larry Wall

Perl may be copied only under the terms of either the Artistic License or the
GNU General Public License, which may be found in the Perl 5 source kit.

Complete documentation for Perl, including FAQ lists, should be found on
this system using "man perl" or "perldoc perl".  If you have access to the
Internet, point your browser at http://www.perl.org/, the Perl Home Page.
\end{minted}

Если вместо этого выводится ошибка <<Command not found>> или какая-то другая, то с установкой, что то пошло не так. Возможно, интерпретатор находится за пределами возможных путей из переменной окружения среды PATH.

\section{Базовый синтаксис языка Perl}
Синтаксис языка Perl близок к синтаксису языка C, а также в нем реализованы идеи и конструкции из shell, awk (популярная утилита для работы с табличными данными), sed и так далее.

\subsection{Простые конструкции в Perl}
Каждая программа на Perl представляет собой последовательность утверждений (statement). Комментарии в языке Perl начинаются с символа \verb|#|.
\begin{minted}{perl}
$a = 42;
say "test";
eval { ... };
do { ... };
my $var;
# Комментарий
\end{minted}
Каждое утверждение возвращает некоторое значение.
\begin{minted}{perl}
# Простые конструкции возвращают значение
$a = 42;        # => 42
say "test";     # => 1
eval { 7 };     # => 7
do { 1; 2; 3 }; # => 3
my $var;        # => undef
\end{minted}
Если утверждение --- это вызов функции, то возвращается значение, которое было возвращено функцией. Если утверждение --- это присвоение переменной некоторого значения, то возвращается значение этой переменной.

\subsection{Инициализация переменных. Режим strict}
Следует отметить, что Perl --- нетипизированный язык, любой переменной можно присвоить значение любого типа. Одна и та же переменная в различных участках программы может принимать значения разных типов.

Одна из особенностей Perl заключается в том, что новые переменные инициализируются автоматически в момент первого использования. Это было очень удобно для создания программ-однострочников, но создавало массу проблем в крупных проектах, когда при опечатке в имени какой-либо переменной Perl просто создавал еще одну переменную при этом менялась логика работы программы, но ошибка оставалась незамеченной и найти её было крайне сложно.

Поэтому в Perl появилась специальная директива strict, которая включает так называемый строгий режим. В этом режиме все переменные должны быть явно инициализированы до их первого использования. В ином случае на этапе компиляции кода выводится исключение о том, что была попытка использовать не объявленную переменную. Инициализация переменной происходит с помощью функции my. Хорошим тоном считается писать все программы (длиной более строчки) в режиме strict.

\subsection{Блоки. Область видимости переменной}
Блоком (scope) называется все, что заключено в фигурные скобки:
\begin{minted}{perl}
{
statement;
statement;
...
}
\end{minted}
Переменные, объявленные внутри такого блока, не видны за его пределами. Это свойство позволяет локализовывать части кода.


\subsection{Управляющие конструкции}
В Perl доступны следующие управляющие конструкции:
\begin{itemize}
  \item \textbf{Условия:} \verb|if|, \verb|unless|, \verb|elsif|, \verb|else|
  \begin{minted}{perl}
    if     ( EXPR ) { ... }
    elsif  ( EXPR ) { ... }
    else            { ... }
  \end{minted}
  С помощью \verb|else| и \verb|elsif| (именно так, а НЕ \verb|elseif| или \verb|else if|) можно указать, что должно быть выполнено, если условие не выполнено. Команда \verb|elsif|, в отличие от \verb|else|, сначала проверяет другое условие. Если условие выполняется, то исполняется код в фигурных скобках, а в ином случае программа переходит к следующему \verb|elsif| или  \verb|else|, если они есть.

  Такой же синтаксис и у команды \verb|unless|:
  \begin{minted}{perl}
      unless ( EXPR ) { ... }
  \end{minted}
  В этом случае код в фигурных скобках выполняется только, если выполнено условие в круглых скобках. При этом плохим тоном считается использование конструкций такого вида:
  \begin{minted}{perl}
      unless ( EXPR ) { ... }
      elsif  ( EXPR ) { ... }
      else            { ... }
  \end{minted}

  \item \textbf{Циклы:} \verb|while|, \verb|until|, \verb|for|, \verb|foreach|
  \begin{minted}{perl}
  while ( EXPR ) { ... }
  \end{minted}
  Оператор \verb|while| исполняет код в теле цикла, пока выполнено некоторое условие. Использование ключевого слова \verb|continue| позволяет указать блок кода, который будет выполнен всегда после каждой итерации. Например, если был совершен преждевременный выход из цикла, блок \verb|continue| все равно будет выполнен. Оператор \verb|until| имеет похожий синтаксис, но выполняет код в теле цикла, пока условие не выполнено:
  \begin{minted}{perl}
  until ( EXPR ) { ... } continue { ... }
  \end{minted}
  Еще два оператора for и foreach являются синонимами в Perl и позволяют итерировать по значениям целочисленной переменной и по элементам списка:
  \begin{minted}{perl}
  for ( EXPR; EXPR; EXPR ) { ... }
  for ( LIST ) { ... }
  for VAR ( LIST ) { ... } # Объявляется переменная, которая будет доступна в обоих scope.
  for VAR ( LIST ) { ... } continue { ... }
  \end{minted}
  \item \textbf{Выбор:} \verb|given|, \verb|when|
  \item \textbf{Безусловный переход:}  \verb|goto|
\end{itemize}

\subsection{Типы данных в Perl}
Основные типы данных в Perl включают в себя:
\begin{itemize}
  \item \textbf{SCALAR}~--- простая переменная, имя всегда начинается с символа \verb|$|, может содержать одно из следующих значений:
  \begin{itemize}
    \item \textbf{Number} (числовое значение): \verb|$s = 1|, \verb|$s = -1e30|.
      Perl работает с числовыми значениями как с числами (то есть быстро) до тех пор, пока не будет к ним обращения как к строке. В последнем случае числовое значение будет преобразовано в строковое.

    \item \textbf{String} (строковое значение): \verb|$s = "str"|. Строка в Perl всегда обрамлена в кавычки.
    \item \textbf{Reference} (ссылка): разыменовывание ссылки \verb|$r| делается в зависимости от того типа данных, который предпологается, что лежит по ссылке:
      \begin{itemize}
        \item \textbf{Scalar} (\verb|$$r|, \verb|${ $r }|)
        \item \textbf{Array} (\verb|@$r|, \verb|@{ $r }|, \verb|$r->[...]|)
        \item \textbf{Hash} (\verb|%$r|, \verb|%{ $r }|, \verb|$r->{...}|)
        \item \textbf{Function} (\verb|&$r|, \verb|&{$r}|, \verb|$r->(...)|)
        \item \textbf{Filehandle} (\verb|*$r|)
        \item \textbf{Lvalue} (\verb|$$r|, \verb|${ $r }|)
        \item \textbf{Reference} (\verb|$$r|, \verb|${ $r }|)
      \end{itemize}
    Нужно иметь в виду, что, если разыменовать ссылку на хэш как массив, получится каша, а именно: получится массив из поочередно ключей и соответствующих значений.
  \end{itemize}
  \item \textbf{ARRAY} (\verb|@a|, \verb|$a[...]|)
  \item \textbf{HASH} (\verb|%h|, \verb|$h{key}|, \verb|$h{...}|)
\end{itemize}


\subsection{Специальные переменные}
В perl существуют специальные глобальные переменные, которые позволяют существенно упростить написание программ:
\begin{itemize}
  \item \verb|$_|, \verb|$ARG| --- аргумент по умолчанию,
  \item \verb|@_|, \verb|@ARG| --- аргументы функции (как массив).
  \item \verb|$a|, \verb|$b| --- переменные, используемые при сортировке:
  \begin{minted}{perl}
  for (sort { $a <=> $b } @ARGV) {
  say "Arg: $_";
  }
  say "Was run by $ENV{USER}";
  \end{minted}
  Поэтому при написании программы категорически не рекомендуется заводить переменные с такими же именами.
  \item \verb|%ENV| --- переменные окружения (как хэш),
  \item \verb|@ARGV| --- аргументы программы (как массив).
\end{itemize}
Также существуют следующие специальные переменные:
\begin{itemize}
  \item \verb|$"|, \verb|$LIST_SEPARATOR| --- разделитель при интерполяции в кавычках.
  \item \verb|$,|, \verb|$OUTPUT_FIELD_SEPARATOR| --- разделитель между элементами списка при выводе (при выводе на экран массива его элементы будут разделены этим символом)
  \item \verb|$/|, \verb|$INPUT_RECORD_SEPARATOR| --- разделитель входного потока для readline
  \item \verb|$\|, \verb|$OUTPUT_RECORD_SEPARATOR| --- разделитель выходного потока для print
  \item \verb|$.|, \verb|$INPUT_LINE_NUMBER|
\end{itemize}
Пример использования:
\begin{minted}{perl}
$" = "."; # $LIST_SEPARATOR
$, = ";"; # $OUTPUT_FIELD_SEPARATOR
$\ = "\n\n"; # $OUTPUT_RECORD_SEPARATOR
while (<>) {
    chomp;
    @a = split /\s+/, $_;
    say "$. @a",@a;
    }
\end{minted}
Есть еще ряд специальных переменных:
\begin{itemize}
  \item \verb|$!|,  \verb|ERRNO|~--- переменная в которую записываются ошибки при открытии файлов.
  \item \verb|$<|,  \verb|UID|~--- ID пользователя, запустившего программу.
  \item \verb|$$|,  \verb|PID|~--- PID процесса
  \item \verb|$0|,  \verb|PROGRAM_NAME|~--- Имя программы
  \item \verb|$^X|,  \verb|EXECUTABLE_NAME|~--- название запущенного бинарного файла
  \item \verb|$^O|,  \verb|OSNAME|~--- имя используемой операционной системы
  \item \verb|$^V|,  \verb|PERL_VERSION|~--- версия perl
\end{itemize}

Например:
\begin{minted}{perl}
say "I'm $^X, $^V, on $^O";
say "Script: $0 (@ARGV);";
say "Pid $$ by uid $<";
open my $f, '<','/etc/shadow'
or die "No shadow: $!\n";
\end{minted}

\begin{minted}{bash}
/usr/bin/perl sample.pl -test
\end{minted}

\begin{verbatim}
# I'm /usr/bin/perl, v5.18.2, on darwin
# Script: sample.pl (-test);
# Pid 70032 by uid 502
# No shadow: No such file or directory
\end{verbatim}

\section{Запуск однострочных скриптов}
Perl в начале своего развития использовался для запуска и исполнения простейших однострочных скриптов. Также, за исключением небольшого числа системнозависимых библиотек, Perl на всех операционных системах работает одинаково. Именно поэтому Perl был так популярен среди системных администраторов.

При написании однострочников \verb|use strict;| писать необязательно. Но если вдруг скрипт начал выполнять не то, что задумано, лучше добавить \verb|use strict;|, что упростит отладку.

Запустить однострочник можно с помощью ключа \verb|-e|, после которого идет код однострочника в кавычках:
\begin{minted}{bash}
perl -e 'print "Hello world\n"'
\end{minted}
Использовать всегда следует одинарные кавычки, так как разные шеллы по-разному работают с двойными кавычками, и существует соглашение, что одинарные кавычки --- неинтерполируемые (то есть в них не будут подставляться переменные среды). Обычно однострочная программа, обрабатывает данные приходящие ей в стандартном вводе. Вот простейшая программа которая читает стандартный ввод и распечатывает строки добавляя в начало каждой из них дефис
\begin{minted}{bash}
perl -e 'while(<>){print "- ".$_}'
\end{minted}
Чтобы постоянно не использовать конструкцию \verb|while(<>){}| при выполнении действия над каждой строчкой в файле, существует ключ \verb|-n|. Идентичная запись той же программы:
\begin{minted}{bash}
perl -ne 'print "- ".$_'
\end{minted}
Внутри цикла \verb|while(<>){}| переменная \verb|$_| будет содержать цельную строку вместе с символом \verb|\n|. Что бы не использовать команду chomp (отрезающую в конце перенос строки) можно использовать флаг \verb|-l|, который:
\begin{itemize}
  \item устанавливает переменную \verb|$\| (разделитель, который выведен после каждого выполнения команды print)
  \item устанавливает переменную \verb|$/| (разделитель по которому будет делится входящий поток, отдельно его можно выставить с помощью флага \verb|-0|)
  \item удаляет из строки \verb|$_| последний перенос строки (при совместном использовании с флагом \verb|-n|)
\end{itemize}
Например, прочитать файл в котором записи разделены через \verb|";"| и вывести каждую запись на новой строке можно так:
\begin{minted}{bash}
perl -nl00120073 -e 'print $_'
\end{minted}
Все строки из файла записать через \verb|";"| можно так:
\begin{minted}{bash}
perl -nl00730012 -e 'print $_'
\end{minted}
Флаг \verb|-p| делает тоже самое, что и флаг \verb|-n|, только в каждую итерацию цикла добавит еще вывод переменной \verb|$_|:
\begin{minted}{bash}
perl -nl00120073 -e ''
\end{minted}
\begin{minted}{bash}
perl -pl00730012 -e ''
\end{minted}
Детальнее можно посмотреть тут: perldoc \perldoc{perlvar}.

Для парсинга более сложных структур файлов (например, когда в каждой строке есть записи разделенные определённым разделителем), можно использовать флаг  \verb|-a| совместно с \verb|-F|:
\begin{itemize}
  \item Флаг \verb|-a| добавляет функцию разделяющую входную строку на части и складывает в спецмассив \verb|@F|
  \item \verb|-F| устанавливает разделитель (по умолчанию это пробел)
\end{itemize}
Например, с помощью следующего кода можно прочитать файл-таблицу, в которой каждая строка представляет собой набор полей разделенных \verb|";"|, проверить третью колонку на наличие там 1 и при выполнении условия вывести значение из 2 колонки:
\begin{minted}{perl}
perl -lnaF';' -e 'if( $F[2] == 1 ){ print $F[1] };'
\end{minted}
В этом примере флаг \verb|-l| необходим, чтобы выполнять скрипт для каждой строки файла.

\subsection{Система CPAN}
Perl завоевал к себе доверие, за счет того, что был портирован под всевозможные платформы и системы, а также за счет большой системы CPAN. CPAN --- архив модулей, написанных на языке программирования Perl.
Все версии модулей под все версии perl проходят тестирование (разработчик модуля должен сам должен предоставить тесты) автоматической системой на CPAN. Эта система запускает модуль на каждой версии perl и отмечает результат выполнения тестов в специальной таблице. Часто при крупных обновлениях perl некоторые модули перестают работать, поэтому при выборе модуля для работы эта таблица может оказаться полезной.

\begin{figure}[H] \centering
  \includegraphics[width=10cm]{lectures/L1/matrixcpantesters.png}
\end{figure}\noindent

Подключить модули (чтобы иметь возможность пользоваться функциями этого модуля) можно опцией \verb|-M|, например:
\begin{minted}{perl}
perl -MJSON::XS -e 'print JSON::XS::encode_json({var1 => 1, var2 => 2})'
\end{minted}
Perl поддерживает еще много других ключей, но представленных должно быть достаточно для начала изучения языка.

\section{Средства отладки в Perl}
\subsection{Доступ к кодам операции}
В perl 5 (начиная с версии 5.005) был обеспечен доступ к компилятору. По умолчанию Perl переводит исходный код в коды операции и исполняет их. Если вы не хотите выполнять свою программу, а хотите проанализировать коды операции, то можно написать собственный модуль или воспользоваться существующим. Для таких модулей было выделено пространство имен \verb|B::|). Доступ к компилятору организован, через модуль \verb|O|, таким образом для передачи кодов операций в некоторый модуль для их анализа, следует использовать модуль \verb|O|, передавая ему имя модуля для получения кодов операций:
\begin{minted}{perl}
perl -MO=Backend
\end{minted}
Указанный модуль может вывести коды операций на экран, проанализировать быстродействие. Эта функция будет особенно полезна при отладке. Например, \verb|B::Concise| --- модуль, позволяющий вывести коды операции как есть.

\subsection{Модуль Deparse}
\verb|B::Deparse| --- модуль, который превращает коды операции после компилятора обратно в код на perl, то есть декомпилирует коды операций. У этого модуля есть множество опций, позволяющих менять его поведение.

Работу модуля можно продемонстрировать на программе
\begin{minted}{perl}
perl -pl00730012 -e ''
\end{minted}
запуская perl с подключенным модулем B::Deparse:
\begin{minted}{perl}
perl -MO=Deparse -pl00730012 -e ''
\end{minted}
На выходе получается:
\begin{minted}{perl}
BEGIN { $/ = "\n"; $\ = ";"; }
LINE: while (defined($_ = <argv>) ){
    chomp $_;
}
continue {
    die "-p destination: $!\n" unless print $_;
}
-e syntax OK
\end{minted}
У модуля B::Deparse есть свои удобные ключи:
\begin{itemize}
  \item \verb|-l| добавит комментарии с ссылками на строки исходного файла
  \item \verb|-p| расставит скобки и тем самым покажет приоритетность выполнения команд
  \item \verb|-q| развернёт интерполируемые строки, то есть представит их как конкатенацию неинтерполируемых строк и переменных. Интерполирование строк --- это поиск внутри строчек, которые обрамлены в двойные кавычки, имен переменных и замена их на соответствующие им значения:
  \begin{minted}{perl}
  "Hello $name" = 'Hello ' . $name != 'Hello $name'
  \end{minted}
  Строки в одинарных кавычках не интерполируются.
  \item Ключ \verb|-sС| определяет стиль вывода кода.
  \item Ключ \verb|-siNUMBER| определяет количество пробелов в отступе
  \item Ключ \verb|-siT| позволяет использовать символ табуляции
  \item Ключ \verb|-xNUMBER| определяет уровень развертывания кода
\end{itemize}
\verb|B::Deparse| можно использовать, как обычный модуль, передавая ему ссылку на функцию, код которой хочется посмотреть (в том числе, если доступна только ссылка на функцию):
\begin{minted}{perl}
use B::Deparse;
sub func {
    print 'Hello world!!!'
};

my $deparse = B::Deparse->new("-p", "-sC");
$body = $deparse->coderef2text(\&func);

print $body;
\end{minted}
После выполнения такой программы:
\begin{minted}{perl}
{
print('Hello world!!!');
}
\end{minted}

При уровне отладки большем чем 3 все циклы for будут развёрнуты в while. Код:
\begin{minted}{bash}
perl -MO=Deparse,x3 -e 'for ($i = 0; $i < 10; ++$i) {print $i;}'
\end{minted}
Превратится в
\begin{minted}{perl}
$i = 0;
while ($i < 10) {
    print $i;
} continue {
    ++$i
}
\end{minted}
\subsection{Data::Dumper}
\verb|Data::Dumper| - модуль, который поможет выводить на экран в развернутом виде сложные структуры данных. Например:
\begin{minted}{perl}
use Data::Dumper;
my $foo = [{a => 1, b => 2},{c => 3, d => 4}];
print Dumper($foo);
\end{minted}
Вот так красиво демонстрирует эту переменную \verb|Data::Dumper|:
\begin{minted}{perl}
$VAR1 = [
    {
      'b' => 2,
      'a' => 1
    },
    {
      'c' => 3,
      'd' => 4
    }
];
\end{minted}
У \verb|Data::Dumper| есть множество опций, которые позволяют управлять стилем вывода: табуляцией, переносом строк и так далее.

\subsection{Модуль DDP (Data::Printer)}
Есть альтернативный модуль для просмотра структур и объектов. Его обычно используют для отладки приложений. У этого модуля не меньше настроек, чем у Data::Dumper, но его проще использовать, например:
\begin{minted}{perl}
use DDP;
my $foo = {a=>1, b=> 2, c=> [1,2,3]};
p $foo;
\end{minted}
Вывод следующий:
\begin{minted}{perl}
\ {
    a   1,
    b   2,
    c   [
        [0] 1,
        [1] 2,
        [2] 3
    ]
}
\end{minted}
Модуль DDP в основном отличается от Data::Dumper следующим:
\begin{itemize}
  \item Автор этого модуля позаботился о цветовой разметке выводимых данных, для удобства чтения.
  \item Так же у этого модуля более расширенный дамп объектов:
  \begin{minted}{perl}
  \ SomeClass  {
    Parents       Moose::Object
    Linear @ISA   SomeClass, Moose::Object
    public methods (3) : bar, foo, meta
    private methods (0)
    internals: {
      _something => 42,
    }
  }
  \end{minted}
  \item Отсутствует сериализация.
\end{itemize}

\subsection{Использование отладчика}
Для начала работы с дебагером рекомендуется прочитать документацию \perldoc{perldebtut}. Отладка программ подразумевает построчное их выполнение с возможностью просмотра состояния переменных между ними.

Запуск отладчика выполняется добавлением ключа \verb|-d| при запуске интерпретатора:
\begin{minted}{perl}
perl -d myscript.pl
\end{minted}
Для того, что бы отладчик запустился скрипт не должен содержать синтаксических ошибок и должен нормально компилироваться через \verb|perl -c|. После запуска отладчика на экране появится:
\begin{minted}{bash}
Loading DB routines from perl5db.pl version 1.44
Editor support available.

Enter h or 'h h' for help, or 'perldoc perldebug' for more help.

DB<1>
\end{minted}
Далее отладчик ждёт команд. Некоторые команды отладчика для просмотра кода и значения переменных:
\begin{verbatim}
- l посмотреть код. Параметра можно указать номер строки, номер строки + интервал, диапазон строк, название функции. На выход вы получаете запрошенный код
- - посмотреть предыдущий код относительно текущей строки
- v посмотреть код вокруг указанной строки
- / поиск по коду в прямом направлении на вход эта команда принимает регулярное выражение, если ничего не передавать то отладчик продолжит поиск по предыдущему запросу
- ? поиск по коду в обратном направлении
- f загрузка файла для просмотра, на вход принимает имя файла
- . вернуть указатель на текущую позицию выполнения кода
- m $obj показать все методы объекта
- M показать список всех загруженных модулей
- S список всех доступных функций в данной точке
- [X|V] [Package] [str|~re] список переменных. Можно передать имя пакета внутри которого интересуют переменные, название переменно или регулярное выражение для названия переменной
\end{verbatim}
Команды отладчика для выполнения кода:
\begin{verbatim}
- p выполнить перл выражение и показать результат
- n шаг вперёд без захода в процедуру
- s шаг вперёд с заходом в процедуру
- T стек вызовов в данной точке
- ! повторить предыдущую команду. На вход можно передать номер команды в истории которую надо повторить
- source file - выполнить команды из файла
- с продолжить выполнение программы. Если параметром указать номер строки или имя функции, то отладчик продолжит выполнение до указанного места
- r продолжить выполнение скрипта до выхода из подпрограммы
- q выход из отладчика
\end{verbatim}
Точки останова, действия, точки наблюдения
\begin{verbatim}
- b < line|sub > [условие] - установить точку останова на указанную строку или функцию при выполнении условия
- B < ln|* > - снять точку останова
- a строка действие [условие] - установить действие которое сработает достигнув определённой строки
- A < line|* > - удалить действие
- w $var - установить наблюдение за переменной
- W $var|* - снять наблюдение за переменной
- L [a|b|w] - вывести список точек останова, действий, наблюдений за переменными
- R - начать скрипт заново оставив все точки останова, действия, наблюдения
\end{verbatim}

\subsection{Использование отладчика: пример}
В качестве примера будет показана отладка следующего скрипта:
\begin{minted}{perl}
use strict;

my $ret = 0;

foreach ( my $i = 0 ; $i < 50 ; $i++ ) {
    if ( $ret > $i ) {
        $ret -= $i;
    }
    else {
        $ret += $i;
    }
}
print $ret;
\end{minted}
Запуск отладки:
\begin{minted}{bash}
perl -d mydebug.pl
\end{minted}
Отладчик запустился и выдал приглашение:
\begin{minted}{bash}
Loading DB routines from perl5db.pl version 1.44
Editor support available.

Enter h or 'h h' for help, or 'perldoc perldebug' for more help.

main::(mydebug.pl:3):   my $ret = 0;

DB< 1 >
\end{minted}
Команда \verb|l| позволяет просмотреть код с указанием текущей позиции:
\begin{minted}{perl}
3==>    my $ret = 0;
4:      my $cnt_add = 0;
5:      my $cnt_sub = 0;
6
7:      foreach(my $i = 0; $i < 50; $i++){
8:              if($ret>$i){
9:                      $ret -= $i;
10              }
11              else {
12:                     $ret += $i;
\end{minted}
Установка точки останова (команда \verb|b|) на 8ой строчке при достижении 10 итерации цикла:
\begin{minted}{bash}
b 8 $i == 9
\end{minted}
Запуск программы (команда \verb|c|)
\begin{minted}{bash}
c
\end{minted}
Когда выполнится условие \verb|$i == 9|, сработает точка останова и на экране появится сообщение:
\begin{minted}{bash}
DB< 7 >
c
main::(mydebug.pl:8):           if($ret > $i){
\end{minted}
Вывод значения переменной \verb|$i|:
\begin{minted}{bash}
DB< 7 > print $i
9
\end{minted}
Отслеживать переменные можно командой \verb|$w|:
\begin{minted}{bash}
w $ret
\end{minted}
После этого выполнив 1 строчку (команда \verb|n|):
\begin{minted}{bash}
DB< 9 >
n
Watchpoint 0:   $ret changed:
old value:  ''
new value:  '16'
main::(mydebug.pl:9):                   $ret -= $i;

DB< 9 >
\end{minted}
Дальнейшая отладка происходит по тем же принципам.

%\include{lectures/L2/L2}
%\include{lectures/L2/L2-bonus}
%\setcounter{chapter}{2}
\chapter{Модульность и повторное использование}
Данная лекция посвящена модульности в языке Perl. Знания базового синтаксиса языка программирования недостаточно, чтобы писать сложные законченные программные продукты, поскольку любой полноценный проект состоит из множества модулей. Построение сложной иерархии проекта с технической и логической точек зрения является темой данной лекции.

\section{Команды типа <<\texttt{include}>>} %1 (1:21)

\subsection{Функция \texttt{eval}}
В языке C команда \verb|include| позволяет подключить другой файл с кодом с помощью <<механической>> подстановки его содержимого. Эта команда позволяет разбивать сложные проекты на несколько файлов.
Похожие команды есть во многих других языках программирования.

Самый простой способ исполнить некоторый код в языке Perl~--- использование функции \verb|eval|. Если передать этой функции строку с кодом, то этот код будет исполнен.

Важной особенностью является то, что \verb|eval| создает свою область видимости, а следовательно, локальные переменные, объявленные с помощью \verb|my|, будут ограничены функцией \verb|eval|. Это можно заметить в приведенном выше примере: переменная \verb|$y| объявляется с помощью \verb|my| и существует только внутри \verb|eval|, а переменная \verb|$u|, объявленная вне \verb|eval|, внутри \verb|eval| изменяет свое значение.

Этот способ прост, но требует выполнения множества дополнительных действий вручную. К таким действиям относятся, например, чтение файлов и обработка ошибок.

\subsection{Функция \texttt{do}} %3 (3:05)
\verb|do|~--- более совершенная версия \verb|eval|. Не следует путать этот \verb|do| с тем \verb|do|, который используется для создания циклов.
Функция \verb|do| принимает имя файла, содержимое которого она
считывает и исполняет с помощью \verb|eval|.

Функция \verb|do|, как и \verb|eval|, создает свою область видимости.
Однако \verb|do| практически не используется в сложных проектах, так как существует более высокоуровневая функция \verb|require|.

\subsection{Функция \texttt{require}} %4 (3:58)
Функция \verb|require|~--- более совершенная высокоуровневая форма \verb|do|.
Ей также нужно передать имя файла, после чего произойдет импорт и последующее исполнение кода из файла.
Однако функция \verb|require| имеет некоторые особенности.

Во-первых, \verb|require| поддерживает синтаксис с использованием двойного двоеточия, что позволяет абстрагироваться от реальных имен файлов и давать модулям названия.
В следующем примере первым вызовом \verb|require| будет загружен файл с расширением \verb|.pl|,
а вторым вызовом~--- файл с расширением \verb|.pm| (perl module):

Во-вторых, функция \verb|require| проверяет, что код модуля после выполнения возвращает истинное значение. Поэтому основная масса модулей заканчивается строчкой <<\verb|1;|>>:
\inputminted
  [frame=leftline,linenos,firstline=1]
  {perl}
  {lectures/L3/L3-example-include-require-sqr.pl}

Такой подход позволяет гарантировать, что последнее выражение в модуле истинно и функция \verb|require| сочтет такой модуль успешно загруженным.
В очень малом числе случаев модулю действительно нужно сообщить, успешно ли он загружен~--- в таких случаях используются специальные проверки.

Синтаксис с двойными двоеточиями позволяет указать путь до модуля

Данный код сработает вне зависимости от операционной системы и используемого в ней разделителя каталогов.

Поиск модулей \verb|require| выполняет в каталогах, содержащихся в массиве
\verb|@INC| (на самом деле функция \verb|do| делает то же самое)

Добавить каталог в этот массив (для того, чтобы \verb|require| искал модуль и в нем) можно следующими способами:
\begin{enumerate}
  \item Добавив каталог в переменную окружения \verb|PERL5LIB|:
    \begin{minted}[frame=leftline,linenos]{bash}
$ PERL5LIB=/tmp/lib perl ...
    \end{minted}
  \item Используя ключ \verb|-I| интерпретатора:
    \begin{minted}[frame=leftline,linenos]{bash}
$ perl -I /tmp/lib ...
    \end{minted}
\end{enumerate}
Помимо указанных способов, можно также явно модифицировать массив
\verb|@INC|, но такой подход очень редко оправдан.

На данный момент рассмотрены основные методы подключения модулей.
Существует, однако, ещё один способ, о котором будет сказано позднее.

% ------------------------------------------------------
\section{Блоки фаз} %7:56
В \verb|perl| сушествует возможность указать блок \verb|BEGIN|, который будет исполнен в начале программы вне зависимости от реального расположения внутри программы:
\begin{minted}{perl}
BEGIN {
  require Some::Module;
}

sub test1 {
  return 'test1';

* sub test2 {
*   return 'test2';
*
*   BEGIN {...}
* }
}
\end{minted}
Важной особенностью \verb|perl| является то, что функции объявляются еще до исполнения программы. В данном примере будет сначала выполнен первый блок \verb|BEGIN|, потом объявлены функции $test1$ и $test2$, выполнен, вложенный в функцию $test2$, блок \verb|BEGIN| и только после этого начнется исполнение программы.

Парный блоку \verb|BEGIN|, блок \verb|END| исполняется, наоборот, когда программа завершилась:
\begin{minted}{perl}
open(my $fh, '>', $file);

while (1) {
  # ...
}

END {
  close($fh);
  unlink($file);
}
\end{minted}
Он исполняется последним вне зависимости от расположения в исходном коде программы. Чаще всего блок \verb|END| используется для очистки ресурсов. В данном примере в блоке \verb|END| реализован процесс завершения работы с файлом.

Также в \verb|perl| существуют блоки:
\begin{itemize}
	\item CHECK\{\}
	\item UNITCHECK\{\}
	\item INIT\{\}
\end{itemize}
Такое большое количество разнообразных блоков фаз связано с работой интерпретатора. Использование нужного блока позволяет исполнить требуемый код в нужный момент работы интерпретатора. Эти особенности далее обсуждаться не будут. В реальном коде данные блоки встречаются крайне редко.

Внутри всех блоков присутствует переменная
\[ \$\{ \textasciicircum GLOBAL \_ PHASE \}, \]
в которой хранится название текущей фазы (\verb|INIT|, \verb|UNITCHECK| и т.п.).

\section{Команды типа <<include>> (продолжение)} %10 (11:20)
Использование ключевого слова \verb|use|~--- основной способ подключения модулей в \verb|perl|, которы представляет собой выполнение \verb|require| внутри блока \verb|BEGIN|. Модули подключаются в заданном порядке.
\begin{minted}{perl}
use My_module;     # My_module.pm
use Data::Dumper;  # Data/Dumper.pm
BEGIN { push(@INC, '/tmp/lib'); }
use Local::Module; # Local/Module.pm
\end{minted}
В данном случае сначала будут подключены два модуля, затем выполнен блок \verb|BEGIN|, а после~--- подключен третий модуль.

Как и \verb|require|, \verb|use| умеет понимать литералы с двойными двоеточиями.

Выполнить \verb|use| можно используя ключ интерпретатора $-M$.

\section{Пространства имен} %11 (12:19)

\begin{minted}{perl}
require Some::Module;
function(); # ?

require Another::Module;
another_function(); # ??

require Another::Module2;
another_function(); # again!?
\end{minted}

В \verb|perl| пространства имён (англ. \verb|namespace|) называются пакетами (англ. \verb|package|). С помощью пакетов можно создать отдельную область видимости для функций и переменных так, чтобы они не были доступны извне по свои коротким именам, но доступны по $full$ $qualified$ $name$.

Ключевое слово \verb|package| используется для объявления пакета и все объявленные функции и переменные до конца области видимости будут входить в этот пакет.
Для имен пакетов используется такой же синтаксис с двумя двоеточиями, который встречался ранее. Это сделано не случайно~--- существует соглашение внутри каждого модуля определять пакет с точно таким же именем. Это позволяет удобно организовать код программы.
\begin{minted}{perl}
require Some::Module;
Some::Module::function();

require Another::Module;
Another::Module::another_function();

require Another::Module2;
Another::Module2::another_function(); # np!
\end{minted}

Например, в следующем примере после подключения модуля Local::Multiplier
\begin{minted}{perl}
use Local::Multiplier;

print Local::Multiplier::m3(8); # 24
\end{minted}
имеется возможность использовать функции, объявленные в одноимённом пакете:
\begin{minted}{perl}
package Local::Multiplier;

sub m2 {
  my ($x) = @_;
  return $x * 2;
}

sub m3 {
  my ($x) = @_;
  return $x * 3;
}
\end{minted}
Имена функций отделяются от имени пакета также двойным двоеточием.

Ключевое слово \verb|package| не обязательно указывать в начале файла. Оно может быть использовано в любом месте и помещает переменные и функции в пакет до конца области видимости. Сразу после этого пакет становится доступен. Например:
\begin{minted}{perl}
{
  package Multiplier;
  sub m_4 { return shift() * 4 }
}

print Multiplier::m_4(8); # 32
\end{minted}
 Имя пакета можно получить используя ключевое слово \verb|__PACKAGE__|:
\begin{minted}{perl}
package Some::Module::Lala;

print __PACKAGE__; # Some::Module::Lala
\end{minted}

\section{Переменные пакета}
Переменные пакета объявляются с помощью ключевого слова \verb|our| (а не \verb|my|) и внутри пакета доступны по короткому имени. К переменным пакета можно обращаться всегда и по длинному имени, но это часто не удобно (например, при переименовании пакета пришлось бы переименовывать все его переменные).
\begin{minted}{perl}
{
  package Some;
  my $x = 1;
  our $y = 2; # $Some::y;

  our @array = qw(foo bar baz);
}

print $Some::x; # ''
print $Some::y; # '2'

print join(' ', @Some::array); # 'foo bar baz'
\end{minted}
Ключевое слово \verb|our| может также использоваться по отношению к массивам и хешам. В этом случае массив будет доступен по короткому имени внутри пакета, как будто это локальная переменная. Объявленные таким образом переменные будут все ещё доступны снаружи пакета по полным именам.

%\subsection{my,state} %16 (18:28)
Следует напомнить, что \verb|my| задает переменную в локальной области видимости:
\begin{minted}{perl}
my $x = 4;
{
  my $x = 5;
  print $x; # 5
}
print $x; # 4
\end{minted}
Если локальная переменная имеет то же имя, что и некоторая переменная из более широкой области видимости, то последняя в локальной области видимости оказывается недоступной.

С помощью конструкции
\begin{minted}{perl}
use feature 'state';
\end{minted}
можно включить возможность определять переменные с помощью ключевого слова \verb|state|. Основное отличие от \verb|my| состоит в том, что присваивание переменной значения происходит только раз за все время исполнения программы:
\begin{minted}{perl}
sub test {
  state $x = 42;
  return $x++;
}

printf(
  '%d %d %d %d %d',
  test(), test(), test(), test(), test()
); # 42 43 44 45 46
\end{minted}
В данном примере в функции \verb|state| переменной $\$x$ значение 42 присваивается только при первом ее вызове.

\section{Глобальная область видимости} %17 (20:04)
Глобальная область видимости является пакетом с именем \verb|main| (или имя которого~--- пустая строка). Например, принудительно использовать переменную из глобальной области видимости можно указав в качестве имени пакета \verb|main|:
\begin{minted}{perl}
our $size = 42;

sub print_size {
  print $main::size;
}

package Some;
main::print_size(); # 42
\end{minted}
В некоторых пакетах глобальные переменные определяются в начале файла, а потом используются в пакете именно таким образом.

\section{Передача параметов} %18 (20:52)
Функция \verb|use|, кроме того, что исполняется в \verb|BEGIN|, поддерживает передачу параметров. Для этого после имени модуля указывается список параметров, которые будут переданы загрузчику модуля:
\begin{minted}{perl}
use Local::Module ('param1', 'param2');
use Another::Module qw(param1 param2);
\end{minted}
Другими словами, при исполнении \verb|use| не просто выполняется \verb|require| внутри блока \verb|BEGIN|, а также вызвается метод \verb|import| одноименного пакета (если он есть), которому собственно и передаются параметры:
\begin{minted}{perl}
BEGIN {
  require Module;
  Module->import(LIST);
  # ~ Module::import('Module', LIST);
}
\end{minted}
Это еще одна причина использовать одинаковые имена пакетов и модулей: функция \verb|use| будет искать метод \verb|import| в пакете, имя которого совпадает с именем загружаемого модуля.

Если метод \verb|import| вызывать не требуется вообще, после имени модуля нужно указать пустые скобки. Следует отметить, что указание пустых скобок и отсутствие скобок~--- не одно и тоже. Если скобки отсутствуют, метод \verb|import| вызывается без параметров. Пустые скобки~--- прямое указание на запрет вызова \verb|import|.

\section{Экспорт} %19 (23:17)
Название метода \verb|import| исходит из того, что в начале он использовался, чтобы выборочно импортировать некоторые функции из пакета. Например, в модуле \verb|File::Path|, одном из основных модулей \verb|perl|, есть функции $make\_path$ (создать множество вложенных каталогов) и $remove\_tree$ (удалить множество каталогов). Если включить \verb|use File::Path| и в качестве параметров передать \verb|qw(make_path remove_tree)|, то в текущем пакете (в данном случае \verb|main|) появятся функции $make\_path$ и $remove\_tree$:
\begin{minted}{perl}
package My::Package;

use File::Path qw(make_path remove_tree);

# File::Path::make_path
make_path('foo/bar/baz', '/zug/zwang');
File::Path::make_path('...');
My::Package::make_path('...');

# File::Path::remove_tree
remove_tree('foo/bar/baz', '/zug/zwang');
File::Path::remove_tree('...');
My::Package::remove_tree('...');
\end{minted}
После этого, так как метод \verb|import| был написан соответствующим образом, можно будет обращаться к данным функциям не только по полным именам, но и по коротким. Это самый простой и распространённый способ экспорта функций, поскольку позволяет контролировать информацию о полученных функциях и избегать конфликта имен с функциями из других модулей.

Поскольку все модули используют такой механизм, он реализован в отдельном модуле \verb|Exporter|:
\begin{minted}{perl}
package Local::Multiplier;

use Exporter 'import';
our @EXPORT = qw(m2 m3 m4 m5 m6);

sub m2 { shift() ** 2 }
sub m3 { shift() ** 3 }
sub m4 { shift() ** 4 }
sub m5 { shift() ** 5 }
sub m6 { shift() ** 6 }
\end{minted}
В пакете объявляется массив функций, которые требуется экспортировать, вызывается через \verb|use Exporter|, и, как только где-то будет вызван данный модуль, все функции, указанные в массиве, будут экспортированы.
\begin{minted}{perl}
use Local::Multiplier;

print m3(5); # 125
print Local::Multiplier::m3(5); # 125
\end{minted}

Чтобы иметь возможность выбирать, какие функции будут экспортированы, следует использовать массив $@EXPORT\_OK$:
\begin{minted}{perl}
package Local::Multiplier;

use Exporter 'import';
our @EXPORT_OK = qw(m2 m3 m4 m5 m6);

sub m2 { shift() ** 2 }
sub m3 { shift() ** 3 }
sub m4 { shift() ** 4 }
sub m5 { shift() ** 5 }
sub m6 { shift() ** 6 }
\end{minted}
В нем указываются функции, которые могут быть экспортированы. Какие из них будут экспортированы, выбираются по короткому имени при вызове данного модуля.
\begin{minted}{perl}
use Local::Multiplier qw(m3);

print m3(5); # 125
print Local::Multiplier::m4(5); # 625
\end{minted}

Также модуль \verb|Exporter| поддерживает экспорт групп функций. Для этого необходимо задать хеш $\%EXPORT\_TAGS$:
\begin{minted}{perl}
our %EXPORT_TAGS = (
  odd  => [qw(m3 m5)],
  even => [qw(m2 m4 m6)],
  all  => [qw(m2 m3 m4 m5 m6)],
);
\end{minted}
Чтобы экспортировать из модуля определенную группу функций используется синтаксис с двоеточием:
\begin{minted}{perl}
use Local::Multiplier qw(:odd);

print m3(5);
\end{minted}
К данному моменту уже перечислены основные механизмы подключения модулей.

\section{Загрузка определенной версии пакета}
Perl поддерживает загрузку пакета строго определенной версии. Для этого требуемую версию необходимо указать сразу после имени пакета:
\begin{minted}{perl}
use File::Path 2.00 qw(make_path);
\end{minted}
Версия пакета указывается внутри пакета в специальной переменной $\$VERSION$ следующим образом:
\begin{minted}{perl}
package Local::Module;

our $VERSION = 1.4;
\end{minted}
Если в это переменной будет значение меньше запрашиваемого, будет выведено сообщение об ошибке:
\begin{minted}{perl}
use Local::Module 1.5;
\end{minted}
\begin{minted}{bash}
$ perl -e 'use Data::Dumper 500'
Data::Dumper version 500 required--
this is only version 2.130_02 at -e line 1.
BEGIN failed--compilation aborted at -e line 1.
\end{minted}

На самом деле при проверке версии пакета вызвается метод \verb|VERSION| этого пакета: %24 29:19
\begin{minted}{perl}
use Local::Module 500;
# Local::Module->VERSION(500);
# ~ Local::Module::VERSION('Local::Module', 500);
\end{minted}
Внутри модуля эту функцию можно определить произвольным образом и тем самым задать то, как происходит проверка версии.
\begin{minted}{perl}
package Local::Module;

sub VERSION {
  my ($package, $version) = @_;

  # ...
}
\end{minted}
Такая функциональность требуется крайне редко.

Уже достаточно давно в \verb|perl| присутствует синтаксис для так называемых \verb|version strings|:
\begin{minted}{perl}
use Local::Module v5.11.133;
\end{minted}
Синтаксис состоит в том, что после имени модуля ставится символ <<v>> и сразу за ним несколько чисел, разделенных точками.
\begin{minted}{perl}
v102.111.111; # 'foo'
102.111.111;  # 'foo'
v1.5;
\end{minted}

Такая запись превращается в последовательность символов. Количество символов в последовательности равняется количеству чисел, а каждый символ в строке таков, что его код равен соответствующему числу. Такое преобразование версии в строку позволяет производить корректное лексикографическое сравнение двух версий. При лексикографическом сравнении сначала сравниваются первые символы двух строк, потом вторые и так далее. Так и при сравнении двух номеров версий сначала будут сравниваться старшие номера версий, и, если они совпадают, следующий за старшим и так далее.

Об этом не стоило бы и говорить, если бы не одно с этим связанное недоразумение. Дело в том, что если переменные или ключи хешей похожи на v-string, интерпретатор может принять эту запись за v-string и сделать вышенаписанное преобразование. Это следует иметь в виду и брать v и числа в кавычки, поскольку иначе запись будет неправильно воспринята интерпретатором.

\section{Указание версии интерпретатора}
При использовании команды \verb|use| можно указать номер версии, но не указывать название пакета. В таком случае будет указана требуемая версия интерпретатора \verb|perl|, а также станут доступны все возможности, которые появились в этой версии:
\begin{minted}{perl}
use 5.12.1;
use 5.012_001;

$^V # v5.12.1
$]  # 5.012001
\end{minted}
Версия интерпретатора хранится в переменной $\$\^V$ (в новом формате) и переменной $\$]$ (в старом формате):
\begin{minted}{perl}
use Module v1.1.1;
use 5.10;
\end{minted}
Именно с этой версией и будет сравниваться указанная после \verb|use| версия.

\section{Pragmatic modules} %27 31:35
С помощью \verb|use| можно загружать так называемые \verb|pragmatic modules|. От обычных модулей они отличаются тем, что (условно) влияют на ход интерпретации программы и их имена традиционно начинаются с маленькой буквы. Однако, строго говоря, какой-то конкретной границы между такими и обычными модулями нет.

Чаще всего используются два pragmatic модуля: \verb|strict| и \verb|warnings|.
\begin{minted}{perl}
use strict;
use warnings;
\end{minted}
Модуль \verb|strict| позволяет включать дополнительные ограничения, а модуль \verb|warnings| включает предупреждения.

\subsection{Модуль strict} %28 33:24
Если параметры не указаны, модуль \verb|strict| включает все три доступных типа ограничений. Фактически, \verb|use strict|~--- это \verb|use strict 'refs'|, \verb|use strict 'vars'|, \verb|use strict 'subs'| вместе взятые. В результате этого некоторые опасные возможности языка становятся недоступными для программиста. Код же, который не использует эти возможности, становится более чистым и надежным.

Использование \verb|use strict 'refs'| позволяет избежать следующей нежелательной ситуации. Следует напомнить, что, если разыменовать указатель на переменную, то получается значение этой переменной:
\begin{minted}{perl}
use strict 'refs';

$ref = \$foo;
print $$ref;  # ok
$ref = "foo";
print $$ref;  # runtime error; normally ok
\end{minted}
Если не указано \verb|use strict 'refs'|, то результатом разыменования переменной-строки будет значение переменной, имя которой есть данная строка (строка может, например, быть считана из стандартного ввода или стороннего файла). Это немного странно и \verb|use strict 'refs'| запрещает такое поведение. Однако иногда такое поведение необходимо, поэтому этот режим можно отключить (об этом будет сказано позднее).

%\subsection{use strict 'vars'} %29 (34:43)
С помощью \verb|use strict 'vars'| можно потребовать явной инициализации переменной с помощью ключевых слов \verb|my| или \verb|our|. Если \verb|use strict 'vars'| не использовать, то обращение (без указания \verb|my| или \verb|our|) в начале файла к переменной $\$x$, фактически, будет обращением к переменной $\$main::x$ (к глобальной переменной), а не к локальной переменной, как это, возможно, задумывалось.

% WARNING: Может поменять местами куски?
С помощью \verb|use strict 'subs'| отключается автоматические перевод bareword'ов (слово без кавычек) в строки:
\begin{minted}{perl}
use strict 'vars';
$Module::a;
my  $x = 4;
our $y = 5;
\end{minted}
Например, если \verb|use strict 'subs'| не используется и функция $test$ не определена:
\begin{minted}{perl}
use strict 'subs';
print Dumper [test]; # 'test'
\end{minted}
Если же до этого определить функцию $test$, поведение совершенно меняется:
\begin{minted}{perl}
sub test {
  return 'str';
}
print Dumper [test]; # 'str'
\end{minted}
Подход, в котором то, как будет интерпретироваться bareword, зависит от того, какие функции существуют на момент исполнения кода, является неприемлемым (кроме, может быть, в случае однострочников).

\subsection{Модуль warnings} %37:10
Модуль \verb|warnings| включает отображение предупреждений только в данной области видимости (в отличие от ключа <<w>> интерпретатора):
\begin{minted}{perl}
use warings;
use warnings 'deprecated';
\end{minted}
Использовать модуль \verb|warnings| более правильно, так как он не включает предупреждения в модулях, где предупреждения могли быть сознательно проигнорированы автором модуля.
\begin{minted}{bash}
$ perl -e 'use warnings; print(5+"a")'
Argument "a" isn't numeric in addition (+) at -e line 1.
\end{minted}

Другой модуль \verb|diagnostics| аналогичен модулю \verb|warnings|, но также выводит подробную инфомацию по каждому предупреждению:
\begin{minted}{bash}
$ perl -e 'use diagnostics; print(5+"a")'
Argument "a" isn't numeric in addition (+) at -e line 1 (#1)
    (W numeric) The indicated string was fed as an argument to an operator
    that expected a numeric value instead.  If you're fortunate the message
    will identify which operator was so unfortunate.
\end{minted}
Это может быть особенно полезно новичкам в \verb|perl|. Использовать же его в production не стоит, так как человек который будет разбираться с предупреждением сможет самостоятельно найти помощь по этой ошибке в интернете.

\subsection{Модули lib и FindBin} %38:30
Модуль \verb|lib| позволяет добавить путь к массиву $@INC$, который содержит директории, в которых будут будет производиться поиск модулей. Вместо того, чтобы вручную добавлять путь с помощью команды \verb|unshift| в блоке \verb|BEGIN|:
\begin{minted}{perl}
use lib qw(/tmp/lib);

BEGIN { unshift(@INC, '/tmp/lib') }
\end{minted}
можно просто воспользоваться этим модулем.

В связке с этим модулем используется модуль \verb|FindBin|, который позволяет сохранить путь к текущему бинарному файлу в некоторой переменной. После этого можно указывать в модуле \verb|lib| путь относительно пути к бинарному файлу:
\begin{minted}{perl}
use FindBin '$Bin';
use lib "$Bin/../lib";
\end{minted}
Обычно так работают standalone-программы.

\subsection{Модуль feature} %40 Опечатка на слайде
Модуль \verb|feautre| позволяет подключить возможность, добавленную в поздних версиях \verb|perl| и которая не была сделана возможностью по умолчанию, например, чтобы избежать конфликта имен. Следующий код подключает функцию \verb|say|, которая отличается от \verb|print| тем, что дополнительно делает перевод строки:
\begin{minted}{perl}
use feature qw(say);

say 'New line follows this';
\end{minted}
Если программист уже определил функцию \verb|say|, то добавление этой функции по умолчанию приведет к конфликту имен.

\subsection{Модуль brignum} %40:13
Модули \verb|bigint| и \verb|bigrat| позволяют отключить встроенное ограничение на длину вычисляемого значения для целых и рациональных чисел соответственно. Например, \verb|bigint| отключает округление при больших значениях целочисленной переменной:
\begin{minted}{perl}
use bignum;
use bigint;
use bigrat;
\end{minted}
\begin{minted}{bash}
$ perl -E 'use bigint; say 500**50'
888178419700125232338905334472656250000000000000000000000000000000000000000000000000000000000000000000000000000000000000000000000000000

$ perl -E 'say 500**50'
8.88178419700125e+134
\end{minted}
Модуль \verb|bignum| подключает оба модуля сразу.

\subsection{Отключение модулей} %43
С помощью \verb|no|, антипода \verb|use|, можно отключить в данный момент не нужные модули. В этом случае вместо метода \verb|import| (в \verb|use|) используется метод \verb|unimport|.
\begin{minted}{perl}
no Local::Module LIST;

# Local::Module->unimport(LIST);
\end{minted}
С помощью \verb|no| можно отключить возможности, которые были добавлены в современных версиях \verb|perl|:
\begin{minted}{perl}
no 5.010;
\end{minted}
Также с помощью \verb|no| можно отключить pragmatic модули, в частности \verb|strict| и \verb|feature|:
\begin{minted}{perl}
no strict;
no feature;
\end{minted}
Обычно эти возможности используются, чтобы локально (в отдельной области видимости) выключить одно из ограничений, накладываемое \verb|strict|, и сделать то, что это ограничение запрещает. После закрывающей фигурной скобки все локально выключенные ограничения вновь будут в силе. Такой подход позволяет использовать потенциально опасные операции только в рамках локальной области видимости и осознанно.


\section{Внутренние механизмы работы perl}
\subsection{Symbol Tables} %42:30
%\subsection{Typeglob}
В \verb|perl| для каждого загруженного пакета создается специальный служебный хеш. Он имеет имя, которое состоит из символа процента, затем имени пакета и двойного двоеточия за ним. Например, если был загружен модуль \verb|Data::Dumper|, то станет доступным хеш \verb|\%Data::Dumper::|.
\begin{minted}{perl}
{
  package Some::Package;
  our $var = 500;
  our @var = (1,2,3);
  our %func = (1 => 2, 3 => 4);
  sub func { return 400 }
}
\end{minted}
Внутри этого хеша можно увидеть так называемую символическую таблицу. Для модуля:
\begin{minted}{perl}
use Data::Dumper;
print Dumper \%Some::Package::;
\end{minted}
соответствующая символическая таблица имеет вид:
\begin{minted}{perl}
$VAR1 = {
          'var' => *Some::Package::var,
          'func' => *Some::Package::func
        };
\end{minted}

% TODO Путается

\subsection{Typeglob} %49 44:30

% TODO Путается

\subsection{Функция caller} %50
Встроенная функция \verb|caller| позволяет получить данные из стека вызовов. Если эта функция была вызвана без параметров, то она вернет название пакета, откуда была вызвана текущая функция, соответствующие имя файла и номер строчки:
\begin{minted}{perl}
# 0         1          2
($package, $filename, $line) = caller();
\end{minted}
В качестве параметра можно указать глубину. В этом случае \verb|caller| вернет гораздо больше информации, в том числе, в каком контексте была вызвана функция и так далее:
\begin{minted}{perl}
(
	$package,    $filename,   $line,
	$subroutine, $hasargs,    $wantarray,
	$evaltext,   $is_require, $hints,
	$bitmask,    $hinthash
) = caller($i);
\end{minted}
\verb|Export| работает именно так: через функцию \verb|caller| узнает имя пакета, откуда он был вызван, а затем с помощью операций над таблицами-символами создаёт нужную функцию в этом пакете.

\subsection{Перехват обращения к несуществующей функции} %51 :48:19
В \verb|perl|, как и во многих других современных интерпретируемых языках, есть способ перехватывать обращения к несуществующим функциям. До того, как будет брошено исключение о том, что запрашиваемой функции нет, будет предпринята попытка вызвать функцию \verb|AUTOLOAD| из этого пакета:
\begin{minted}{perl}
\end{minted}
В переменной пакета $\$AUTOLOAD$ будет лежать имя той функции, которая пыталась быть вызвана. Стоит отметить, что в качестве параметров функции \verb|AUTOLOAD| передаются параметры вызываемой функции.

Это позволяет объявлять не одну функцию, а сразу класс функций. Для тех, кто знаком с интерпретируемыми языками, это механизм уже может быть знаком. Например, в Ruby это называется missing method.

\subsection{Ключевое слово local} %52 50:02
Помимо \verb|my|, \verb|state| и \verb|our|, есть еще похожее на них ключевое слово \verb|local|, которое, однако, не имеет с ними ничего общего. В \verb|perl| существует возможность временно присвоить любой переменной некоторое значение до конца области видимости:
\begin{minted}{perl}

{
  package Test;
  our $x = 123;

  sub bark { print $x }
}

Test::bark(); # 123
{
  local $Test::x = 321;
  Test::bark(); # 321
}
Test::bark(); # 123
\end{minted}
Чаще всего это используется в тех случаях, когда требуется временно поменять поведение и восстановить старое поведение после.

В частности, можно временно изменить значения служебных переменных. Поскольку эти переменные используются внутренними механизмами \verb|perl|, их прежнее значение должно быть возвращено. Примером использование данной возможности является переопределение служебной переменной, которая определяет перенос конца строки, чтобы считывать файлы не построчно, а целиком.

На самом деле с помощью \verb|local| можно переопределять не только переменные, но даже ключи в хеше. Более того, существует конструкция \verb|delete local|, которая удалит ключ, но только локально до конца области видимости, а потом вернет его на место. Возможности \verb|local| безграничны, но рекомендуется не злоупотреблять им, потому что в сложных конструкциях его действие может быть не очевидно:
\begin{minted}{perl}
# localization of values
local $foo;
local (@wid, %get);
local $foo = "flurp";
local @oof = @bar;
local $hash{key} = "val";
delete local $hash{key};
local ($cond ? $v1 : $v2);

# localization of symbols
local *FH;
local *merlyn = *randal;

local *merlyn = 'randal';
local *merlyn = \$randal;
\end{minted}
Подробную справку по ключевому слову \verb|local| можно найти в документации.

%\include{lectures/L4/L4}
%\include{lectures/L5/L5}
%\setcounter{chapter}{5}
\chapter{Объектно-ориентированное программирование}
\section{Введение: объекты, методы и атрибуты} % %3 01:59
ООП~--- самая распространенная на сегодня парадигма программирования. Исторически ООП в Perl не было и оно было добавлено позже. Объект~--- данные и методы работы с ними.
В Perl уже существуют структуры данных (массивы, хеши, и их комбинации), а также наборы поведений (пакеты). Остается объединить эти две сущности.

Ключевое слово \verb|bless| именно это и делает~--- превращает структуру данных в объект с методами из некоторого пакета:
\inputminted
    [frame=leftline,framesep=1em,linenos,firstline=1]
    {perl}
    {lectures/L6/L6-example-intro-bless.pl}

Объект можно сделать из любого ref-а (скаляр, массив или хеш),
но, как правило, используются только ссылки на хеш.
Настоятельно рекомендуется использовать именно этот способ.
В хеш можно положить разнообразные данные в отличие от скаляра или массива,
поэтому использование ссылки на хеш не накладывает дополнительных ограничений;

Из объекта можно вызвать метод с помощью стрелочки \verb'->',
за которой следуют имя метода и его аргумент:
\inputminted
    [frame=leftline,framesep=1em,linenos,firstline=1]
    {perl}
    {lectures/L6/L6-example-intro-method.pl}

Реально произойдет следующее: из пакета будет вызвана функция, причем первым аргументом будет передана ссылка на объект, из которого был вызван метод. Таким образом устанавливается обратная связь между методом и объектом.
Все остальные аргументы передаются как и при обычном вызове.
Во многих языках программирования имеется возможность обращения метода к объекту, из которого его вызвали, с помощью ключевого слова. В Perl такое ключевое слово традиционно обозначается \verb|$self|. Несмотря на то, что имеется возможность задать ключевому слову собственное обозначение, большого смысла в этом нет.

Помимо методов объект наделен атрибутами. Если объект был сделан из хеша, то значения в хеше можно считать атрибутами, а ключи~--- именами атрибутов.

В Perl нет ключевого слова \verb|new|. Но для удобства можно определить метод, который будет вести себя похожим образом:
\inputminted
    [frame=leftline,framesep=1em,linenos,firstline=1]
    {perl}
    {lectures/L6/L6-example-intro-new.pl}

В следующем примере метод \verb|new| вызывается из пакета.
Такой метод называется методом класса:
\inputminted
    [frame=leftline,framesep=1em,linenos,firstline=1]
    {perl}
    {lectures/L6/L6-example-intro-class.pl}

В этом случае в качестве первого параметра будет передана строка с именем пакета
(такой аргумент принято называть \verb|$class|, а не \verb|$self|).
Методы класса синтаксически ничем не отличаются от методов объекта, кроме как способом вызова.
Особо следует отметить, что в \verb|$class| могут попадать и имена наследников, если используется наследование.

Существует еще один способ вызова методов~--- после стрелочки явно указать полное имя функции
\footnote, которую нужно вызвать, вместе с именем пакета:
\inputminted
    [frame=leftline,framesep=1em,linenos,firstline=1]
    {perl}
    {lectures/L6/L6-example-intro-fqn.pl}

Важно уточнить, что все объявленные методы~--- функции из пакета.
Они все еще могут быть вызваны через <<\verb|::|>>,
но в этом случае не будет передано имя класса и не будет работать механизм наследования.

В Perl существует возможность создавать объекты похожим на C++ образом:
\begin{minted}[frame=leftline,framesep=1em,linenos,firstline=1]{perl}
new My::Class(1, 2, 3);

My::Class->new(1, 2, 3);
\end{minted}

Для этого был придуман \textbf{indirect method call}, который позволяет вызвать любой метод следующим образом:
\begin{minted}[frame=leftline,framesep=1em,linenos,firstline=1]{perl}
foo $obj(123);  # $obj->foo(123);
\end{minted}

Рекомендуется никогда не использовать этот метод. Есть замечательный пример, почему так делать не стоит:
\inputminted
    [frame=leftline,framesep=1em,linenos,firstline=1]
    {perl}
    {lectures/L6/L6-example-intro-syntax-error.pl}
Этот пример проходит проверку синтаксиса, хотя совершенно не похож на перловый код.
Здесь из класса \verb|error| вызывается метод \verb|Syntax|, для которого
восклицательный знак и \verb|exit 0;| являются параметром:
\inputminted
    [frame=leftline,framesep=1em,linenos,firstline=1]
    {perl}
    {lectures/L6/L6-example-intro-syntax-error-desc.pl}
Поскольку параметры вычисляются до вызова функции, то ещё до того, как произойдет
вызов \verb|'error'->Syntax| вычисляется параметр \verb|exit 0;|,
который и завершает программу с успешным кодом возврата \verb|0|.
Ошибку о том, что в пакете \verb|error| нет метода \verb|Syntax|,
Perl вывести не успевает, так как программа завершает исполнение раньше.

Во всех классах определён метод \verb|can|:
\inputminted
    [frame=leftline,framesep=1em,linenos,firstline=1]
    {perl}
    {lectures/L6/L6-example-intro-can.pl}
Метод \verb|can| позволяет проверить, может ли использоваться метод,
имя которого передано в параметрах.
\inputminted
    [frame=leftline,framesep=1em,linenos,firstline=1]
    {perl}
    {lectures/L6/L6-example-intro-can-bless.pl}
Метод \verb|can| возвращает ref метода в случае, если метод существует:
\inputminted
    [frame=leftline,framesep=1em,linenos,firstline=1]
    {perl}
    {lectures/L6/L6-example-intro-can-ref.pl}
Здесь передается строка \verb|'A'| в качестве параметра,
так как \verb|can| возвращает уже не метод, а функцию.

Метод \verb|can| можно вызывать не только из классов, но и из объектов:
\inputminted
    [frame=leftline,framesep=1em,linenos,firstline=1]
    {perl}
    {lectures/L6/L6-example-intro-can-bless.pl}

\textbf{Filehandle}~--- тоже объект, у которого есть, в том числе, следующие методы:
\begin{itemize}
  \item \verb|autoflush|~--- сброс буфера,
  \item \verb|print|~--- печать
\end{itemize}

\inputminted
    [frame=leftline,framesep=1em,linenos,firstline=1]
    {perl}
    {lectures/L6/L6-example-intro-filehandle.pl}
Соответственно, предустановленные \verb|STDOUT| и \verb|STDERR|~--- тоже объекты,
и из них можно вызывать методы.

На прошлой лекции не было сказано, почему при вызове \verb|use| и \verb|no| в качестве первого аргумента передается имя пакета.
Теперь понятно, что на самом деле вызывается функция из пакета, первым параметром которой является имя пакета.
\inputminted
    [frame=leftline,framesep=1em,linenos,firstline=1]
    {perl}
    {lectures/L6/L6-example-intro-use.pl}


\section{Примеры использования ООП}% 22:54
\subsection{Модуль DBI}
Модуль \textbf{DBI} (database interface) задаёт интерфейс работы с абстрактной базой данных. От него наследуются все реализации движков конкретных систем управления базами данных (СУБД).
Метод \verb|DBI->connect| позволяет задать параметры подключения и возвращает
\textbf{DBH} (database handler):
\inputminted
    [frame=leftline,framesep=1em,linenos,firstline=1]
    {perl}
    {lectures/L6/L6-example-oop-dbi.pl}

Метод \verb|do| объекта \verb|dbh| позволяет выполнять запросы.
Он возвращает указатель на следующий объект, из которого уже можно получить результат.


\subsection{XML::LibXML}%16 23:56
Библиотека \textbf{XML::LibXML}~--- работает с системами библиотек LibXML и используется для парсинга XML:
\inputminted
    [frame=leftline,framesep=1em,linenos,firstline=1]
    {perl}
    {lectures/L6/L6-example-oop-dbi.pl}

Метод \verb|load_xml| парсит XML из строчки в XML-объект, у которого есть методы для работы с XML.
Например, \verb|findnodes| возвращает список nodes определённого типа. В этой библиотеке широко используется наследование:
\begin{verbatim}
             XML::LibXML::Node
              /             \
XML::LibXML::Document  XML::LibXML::Element
\end{verbatim}
% TODO про производительность


\subsection{File::Spec}%17 25:38
Базовый модуль \textbf{File::Spec} содержит метод \verb|catfile|,
который склеивает из фрагментов путь к файлу:
\inputminted
    [frame=leftline,framesep=1em,linenos,firstline=1]
    {perl}
    {lectures/L6/L6-example-oop-file.pl}

\textbf{File::Spec} автоматически убирает дублирующие слеши, а также учитывает особенности используемой операционной системы.


\subsection{JSON} %18 26:37
У модуля JSON помимо необъектного существует объектный интерфейс.
Каждый его метод возвращает одну и ту же ссылку, но меняет при этом внутренние атрибуты.
\inputminted
    [frame=leftline,framesep=1em,linenos,firstline=1]
    {perl}
    {lectures/L6/L6-example-oop-json.pl}


\section{Наследование}%20 28:59
Класс Рысь (Lynx) наследуется от классов собака и кошка (ввиду похожести). В специальном массиве \textbf{@ISA} хранятся родители класса. Это переменная пакета, куда можно положить имена родителей, и пакет будет немедленно наследован от них. Блок \textbf{BEGIN} стоит для того, чтобы наследование происходило до исполнения кода.
\begin{minted}{perl}
{
  package Lynx;

  BEGIN { push(@ISA, 'Dog', 'Cat') }

}
\end{minted}
Альтернативно можно использовать любую из двух прагм:
\begin{minted}{perl}
  use base qw(Dog Cat);
  use parent qw(Dog Cat);
\end{minted}
В новых проектах рекоментуется использовать более современную прагму parent, но если в проекте уже используется base~--- рекомендуется продолжать его использовать.

В perl поддерживается, как можно догадаться, множественное наследование: поиск метода будет производиться у родителей в том порядке как они были указаны при наследовании.

Класс UNIVERSAL является супер-родителем всех остальных классов:
\begin{minted}{perl}
$obj->can('method');

$obj->isa('Animal');
Dog->isa('Animal');

$obj->VERSION(5.12);
\end{minted}
Именно в этом классе определены методы can, VERSION и isa. Метод isa возвращает true, если объект принадлежит классу, имя которого передано в качестве аргумента, или любому его наследнику.

В каждом классе существует специальный псевдокласс SUPER, указывающий на родителя. Так можно у наследника вызвать метод родителя, который был переопределен при наследовании:
\begin{minted}{perl}
sub foo {
  my ($self, %params) = @_;

  $self->SUPER::foo(%params);

  return;
}
\end{minted}
Порядок разрешения методов будет рассмотрен позже.

\section{Method Resolution Order}% 34:32
В момент, когда множественное наследование только появилось, немедленно выяснилось, что порядок, в котором будет происходить поиск метода по дереву, далеко неочевиден.

Существует простой и очевидный алгоритм поиска, который используется в Perl~--- просто подряд перебирать родителей класса, которые, в свою очередь, перебирают своих:
\begin{verbatim}
      Animal
        |
      Pet   Barkable
      /   \   /
      Cat   Dog
      \   /
      Lynx
\end{verbatim}
В результате вызова метода:
\begin{minted}{perl}
Lynx->method();
\end{minted}
поиск будет осуществляться в следующем порядке:
\begin{minted}{perl}
qw(Lynx Cat Pet Animal Dog Barkable);
\end{minted}
У этого подхода есть серьезная проблема: если в классе Pet метод класса был переопределен, он будет использоваться, вне зависимости от того, был ли он переопределен у Dog. Очевидно, такое поведение не подходит.

Чтобы решить эту проблему есть алогритм \textbf{c3}, который не идет наверх, пока не будут опрошены все дочерние классы:
\begin{minted}{perl}
use mro 'c3';
Lynx->method();
qw(Lynx Cat Dog Pet Animal Barkable);
\end{minted}
Такой порядок поиска метода гораздо более полезен. Включить такой способ поиска методов можно с помощью  \textbf{use mro 'c3'}. После этого (дополнительно к псевдопакету SUPER) появится псевдопакет next, который отличается от SUPER тем, что метод будет искаться по алгоритму \textbf{c3}. Следует особо отметить, что при использовании next не нужно повторять имя метода:
\begin{minted}{perl}
package A;
use mro;

sub foo {
  my ($self, $param) = @_;

  $param++;

  return $obj->next::method($param);
}
\end{minted}

\section{Детали} %28 40:16
Встроенная функция \textbf{ref} сообщает то, на что идет ссылка:
\begin{minted}{perl}
use JSON:

ref JSON->new(); # 'JSON'
ref [];          # 'ARRAY'
ref {};          # 'HASH'
ref 0;           # ''
\end{minted}
Для объектов она возвращает имя класса, к которому принадлежит объект. Часто бывает удобнее использовать функцию blessed:
\begin{minted}{perl}
use JSON:
use Scalar::Util 'blessed';

blessed JSON->new(); # 'JSON'
blessed [];          # undef
blessed {};          # undef
blessed 0;           # undef
\end{minted}
Функция blessed на объектах ведет себя как ref, а во всех остальных случаях возвращает undef.

В классах, как уже было сказано, есть метод can, который говорит, есть ли какие-то методы в классе или нет. Если какие-то методы были определены через \textbf{AUTOLOAD}, то для can они будут невидимы:
\begin{minted}{perl}
package A;
our $AUTOLOAD;
sub new {
  my ($class, %params) = @_;
  return bless \%params, $class;
}
sub AUTOLOAD { print $AUTOLOAD }
\end{minted}
\begin{minted}{perl}
A->new()->test(); # test
A->can('anything'); # :(
\end{minted}
Соответственно, если переопределяется AUTOLOAD, необходимо также переопределить can. Порядок поиска AUTOLOAD следующий:
\begin{minted}{perl}
sub UNIVERSAL::AUTOLOAD {}

# Dog->m(); Animal->m(); UNIVERSAL->m();
# Dog->AUTOLOAD(); Animal->AUTOLOAD();
# UNIVERSAL->AUTOLOAD();
\end{minted}
Интересно, что можно переопределить \verb|UNIVERSAL->AUTOLOAD|, который будет перехватывать вызовы всех несущетсвующих функций.

В perl нет конструкторов, <<конструктор>> обычно реализуется вручную как метод класса, обычно имеющий название new (обычно достаточно один раз определить такой конструктор для того класса, от которого будут наследоваться все классы в проекте). Но в perl есть деструкторы.
\begin{minted}{perl}
package A;
sub new {
  my ($class, %params) = @_;
  return bless \%params, $class;
}

sub DESTROY {
  my ($self) = @_;
  print 'D';
}
\end{minted}
Для того, чтобы создать деструктор, необходимо объявить метод DESTROY. Он будет вызываться при уничтожении объекта, когда на него закончатся ссылки.
\begin{minted}{perl}
A->new(); # print 'D'
\end{minted}
В этом примере созданный объект был сразу же уничтожен, поскольку не был сохранен в переменную.

При использовании деструкторов возникают следующие сложности:
\begin{itemize}
  \item Если внутри деструктора сделать die, программа не упадет, но приведет к изменению переменной \verb|$@|.

  \item В деструкторе нужно быть максимально аккуратным при работе с глобальными переменными. Для всех служебных переменных следует использовать local (локализовать), чтобы неявный вызов деструктора не приводил к нежелательному изменению глобальных переменных. Особенно аккуратно нужно быть с использованием регулярных выражений, так как соответствующие им глобальные переменные будут изменены и это может оказать влияние на исполняющийся в данный момент код.

  \item Существует следующая проблема с AUTOLOAD. В документации написано, что если DESTROY (если он не объявлен) попадает в AUTOLOAD. Но начиная с версии 5.18 это поведение было сломано в результате оптимизации производительности: DESTROY теперь не попадает в AUTOLOAD. Но это и не должно стать проблемой, так как хорошим тоном считается объявлять пустой DESTROY, если в классе объявлен AUTOLOAD.

  \item Когда программа завершается, вызываются все деструкторы всех объектов, которые остались в памяти, причем порядок вызова этих объектов не определен. Может получиться ситуация, что в деструкторе какого-то объекта будут обращения к объектам, которые уже уничтожены. Обычно используется следующее решение:
\begin{minted}{perl}
  sub DESTROY {
    my ($self) = @_;
    $self->{handle}->close() if $self->{handle};
  }
\end{minted}
  Также существует следующий хак: \verb|${^GLOBAL_PHASE} eq 'DESTRUCT'|
\end{itemize}

Поскольку атрибуты объектов~--- элементы хеша, и перехватить обращение к ним представляет сложную задачу, имеет смысл работать с ними с помощью специально созданных для этого методов (сеттеры и геттеры). Некоторые модули позволяют делать это автоматически:
\begin{minted}{perl}
package Foo;
use base qw(Class::Accessor);
Foo->follow_best_practice;
Foo->mk_accessors(qw(name role salary));
\end{minted}
Метод \verb|follow_best_practice| приводит к тому, что получившиеся методы будут называться \verb|set_<name>| и \verb|get_<name>|. Если его не вызывать, то метод на каждый атрибут будет один и без параметра он будет работать как getter, а с параметром~--- как сеттер.

Существуют также более быстрые реализации этого пакета:
\begin{minted}{perl}
use base qw(Class::Accessor::Fast);
use base qw(Class::XSAccessor);
\end{minted}
Рекомендуется использовать Class::XSAccessor, так как он написан на Си и работает очень быстро.

\section{Moose-like} %34 52:04
Поскольку реализация ООП в perl неудобна, появился такой продукт как Moose. Этот модуль стал де-факто стандартом реализации ООП в perl. Первое, что позволяет Moose, это добавление <<атрибутов>> в классы:
\begin{minted}{perl}
package Person;

use Moose;

has first_name => (
  is  => 'rw',
  isa => 'Str',
);

has last_name => (
  is  => 'rw',
  isa => 'Str',
);
\end{minted}
По сути это те же самые сеттеры и геттеры с гораздо более изящным и изощренным синтаксисом. При задании атрибута с помощью has необходимо указать в is, будет ли меняться значение атрибута на лету, а также в isa указать тип атрибута. В Moose уже определен свой метод new и переопределять его нельзя:
\begin{minted}{perl}
Person->new(
  first_name => 'Vadim',
  last_name  => 'Pushtaev',
);
\end{minted}
Наследование происходит с помощью ключевого слова extends:
\begin{minted}{perl}
package User;

use Moose;

extends 'Person';

has password => (
  is => 'ro',
  isa => 'Str',
);
\end{minted}
Часто в момент создания объекта необходимо выполнить определенный код. Для этого используется хук внутри конструктора, который исполняет код в \textbf{BUILD}:
\begin{minted}{perl}
has age      => (is => 'ro', isa => 'Int');
has is_adult => (is => 'rw', isa => 'Bool');

sub BUILD {
  my ($self) = @_;

  $self->is_adult($self->age >= 18);

  return;
}
\end{minted}
Есть более изящный метод сделать тоже самое~--- передать ссылку на процедуру при определении атрибута в поле \textbf{default}:
\begin{minted}{perl}
has age      => (is => 'ro', isa => 'Int');
has is_adult => (
  is => 'ro',
  isa => 'Bool',
  lazy => 1,
  default => sub {
    my ($self) = @_;
    return $self->age >= 18;
  }
);
\end{minted}
Здесь также необходимо гарантировать, что \verb|is_adult| будет инициализирована после возраста. Для этого указывается \verb|lazy => 1|: в этом случае \verb|is_adult| будет инициализироваться только тогда, когда понадобится.

Помимо этого можно определить хук в отдельном методе и передать в поле \textbf{builder}:
\begin{minted}{perl}
has age      => (is => 'ro', isa => 'Int');
has is_adult => (
  is => 'ro', isa => 'Bool',
  lazy => 1,  builder => '_build_is_adult',
);

sub _build_is_adult {
  my ($self) = @_;
  return $self->age >= 18;
}
\end{minted}
В отличии от хука с default, метод, на который указывает builder, можно переопределить при наследовании:
\begin{minted}{perl}
package SuperMan;
extends 'Person';
sub _build_is_adult { return 1; }
\end{minted}

Передав в has вместо имени массив имен, можно определить множество атрибутов сразу:
\begin{minted}{perl}
has [qw(
  file_name
  fh
  file_content
  xml_document
  data
)] => (
  lazy_build => 1,
  # ...
);

sub _build_fh           { open(file_name) }
sub _build_file_content { read(fh) }
sub _build_xml_document { parse(file_content) }
sub _build_data         { find(xml_document) }
\end{minted}
С помощью \verb|lazy_build| можно строить цепочки инициализации, где одно поле зависит от другого.

Если есть ряд классов, которые не наследованы от общего родителя, но к ним должно быть добавлено одно и то же поведение. Это поведение называется ролью, которая может быть подмешана к классу:
\begin{minted}{perl}
with 'Role::HasPassword';
\end{minted}
Чтобы объявить роль, необходимо объявить пакет с \verb|use Moose::Role;| вместо \verb|use Moose;|:
\begin{minted}{perl}
package Role::HasPassword;
use Moose::Role;
use Some::Digest;

has password => (
  is => 'ro',
  isa => 'Str',
);

sub password_digest {
  my ($self) = @_;

  return Some::Digest->new($self->password);
}
\end{minted}
Существует механизм делегирования с помощью поля handles:
\begin{minted}{perl}
has doc => (
  is    => 'ro',
  isa   => 'Item',
  handles => [qw(read write size)],
);
\end{minted}
В этом случае методы read, write и size будут вызваны из doc, а не из самого объекта. Существует способ мапить имена:
\begin{minted}{perl}
has last_login => (
  is    => 'rw',
  isa   => 'DateTime',
  handles => { 'date_of_last_login' => 'date' },
);
\end{minted}
И даже использовать регулярные выражения:
\begin{minted}{perl}
{
  handles => qr/^get_(a|b|c)|set_(a|d|e)$/,
  handles => 'Role::Name',
}
\end{minted}

В Moose хуки before и after можно навешивать на поля:
\begin{minted}{perl}
before 'is_adult' => sub { shift->recalculate_age }
\end{minted}
Можно создавать свои типы на основе готовых типов, писать свои проверки и сообщения об ошибках:
\begin{minted}{perl}
subtype 'ModernDateTime'
  => as 'DateTime'
  => where { $_->year() >= 1980 }
  => message { 'The date is not modern enough' };

has 'valid_dates' => (
  is  => 'ro',
  isa => 'ArrayRef[DateTime]',
);
\end{minted}
Даже существует возможность использовать расширения:
\begin{minted}{perl}
package Config;
use MooseX::Singleton; # instead of Moose
has 'cache_dir' => ( ... );
\end{minted}

Так как Moose работает медленно, существует несколько переписанных реализаций модуля:
\begin{itemize}
	\item Moose
    \item Mouse~--- самая быстрая версия, написанная на C++, но имеет, в отличие от Moose, проблемы с расширениями.
    \item Mo~--- работает еще быстрее
    \item M~--- самая легковесная версия
\end{itemize}
Наиболее востребованные версии~--- Mouse и Mo.

\setcounter{chapter}{6}
\chapter{Работа с базами данных}

% https://www.opennet.ru/base/dev/perl_dbi.txt.html
% http://www.mysql.ru/docs/man/Perl_DBI_Class.html

В рамках данной лекции будут обсуждаться взаимодействие с базами данных и основные библиотеки для Perl, которые позволяют осуществлять это взаимодействие.
Рассматриваться будут преимущественно реляционные базы данных.



\section{Основы реляционных баз данных. SQL}
Для начала необходимо кратко изложить основы реляционных баз данных.
Описание в данном разделе не претендует на теоретическую строгость и ведётся с сугубо практической точки зрения.

\subsection{Реляционные базы данных}
\begin{figure}[H] \centering
  \input{lectures/L7/DB.tikz}
  \caption{Схема базы данных}
\end{figure}

Реляционная база данных хранит данные в таблицах. Каждая таблица описывается в виде перечисления своих полей (столбцов таблицы) и записей, которые в ней хранятся.
Например, в представленной выше базе данных
информация о студентах (имя, факультет и группа) хранится в таблице <<student>>,
об учителях (имя и фамилия)~--- в таблице teacher,
о домашних работах (название работы и путь к работе в репозитории)~---
в таблице homework,
об оценках~--- таблице grade.
При этом в таблице с оценками, которые поставлены студентам преподавателями за какие-то домашние работы, содержатся только идентификаторы студента, преподавателя и работы, а также собственно оценки.

На основании первичных данных, хранящихся в этих 4-х таблицах, можно
получать данные более сложной структуры.
Например, можно получить все оценки определённого студента.
Для описания подобного рода запросов существует так называемый \emph{SQL}~---
\emph{structured query language} (язык структурированных запросов).
Это декларативный язык: он описывает то, какие данные нужно получить, а не то,
как получить эти данные.
Следует, однако, заметить, что в различных базах данных реализация языка SQL может отличаться.


\subsection{Примеры запросов на языке SQL}

\begin{minted}{SQL}
SELECT name, surname
FROM users
WHERE age > 18;
\end{minted}

\begin{minted}{SQL}
SELECT balance
FROM account
WHERE user_id = 81858
\end{minted}

\begin{minted}{SQL}
SELECT *
FROM users u JOIN accounts a
    ON u.id = a.user_id
WHERE account.balance > 0
\end{minted}

\subsection{Оператор \texttt{SELECT}}
Простейший запрос на получение данных можно произвести с помощью оператора \verb|SELECT|. Для этого нужно указать в запросе имена желаемых столбцов и имя таблицы, из которой будут получены данные.
\begin{minted}{SQL}
SELECT id, name
FROM students;
\end{minted}
Чтобы вывести все существующие столбцы в качестве списка столбцов нужно указать
астериск (<<звездочку>>):
\begin{minted}{SQL}
SELECT *
FROM students;
\end{minted}
Так можно вывести на экран все строки, находящиеся в таблице <<students>>.
Для каждой строки будут выведены данные из всех имеющихся столбцов.

Ключевое слово \verb|WHERE| позволяет указать условие, которому должны удовлетворять требуемые строки:
\begin{minted}{SQL}
SELECT *
FROM grade
WHERE point > 0;
\end{minted}
Условие может быть достаточно сложным. Например, оператор \verb|LIKE| позволяет проверять, подходит ли строка под указанный шаблон. Следующий запрос выводит список учителей, в имени которых содержится большая буква <<\verb|B|>>:
\begin{minted}{SQL}
SELECT *
FROM teachers
WHERE first_name LIKE '%B%';
\end{minted}

\subsection{Оператор \texttt{JOIN}}
Оператор \verb|JOIN| позволяет выбирать данные из нескольких таблиц, чтобы представить их в виде одного результирующего набора. При этом необходимо явно задавать условие соединения:
\begin{minted}{SQL}
SELECT *
FROM homework
JOIN grade
    ON homework.id = grade.homework_id
\end{minted}

Оператор \verb|JOIN| может быть применён последовательно несколько раз подряд, если необходимо объединить более двух таблиц:
\begin{minted}{SQL}
SELECT *
FROM homework
JOIN grade
    ON homework.id = grade.homework_id
JOIN teacher
    ON teacher.id = grade.teacher_id;
\end{minted}

% TODO про LEFT JOIN

Порядок строк в результате произволен. Отсортировать полученные результаты можно с помощью конструкции \verb|ORDER BY|:
\begin{minted}{SQL}
SELECT *
FROM teachers
ORDER BY first_name;
\end{minted}
Чтобы отсортировать в обратном порядке, следует добавить \verb|DESC| следующим образом:
\begin{minted}{SQL}
SELECT *
FROM teachers
ORDER BY first_name DESC;
\end{minted}
Сортировка в прямом порядке указывается с помощью ключевого слова \verb|ASC|,
которое по стандарту SQL предполагается по умолчанию.
\subsection{Примеры запросов}
% DELETE UPDATE INSERT
% 13-06 CODE

\section{Типы баз данных. Модуль DBI}
% TODO

Модуль DBI описывает интерфейс, по которому должны работать все другие модули для связи с базами данных. Это позволяет унифицировать запросы к совершенно разным базам данных при программировании на Perl. Однако специфика каждой конкретной базы данных не перестаёт играть роль.

\subsection{Метод \texttt{connect}}
Подключиться к базе данных можно с помощью метода \verb|connect| модуля DBI, который в качестве своих аргументов принимает расположение базы данных, имя пользователя и пароль.
В качестве четвёртого аргумента можно передать дополнительные параметры.
В результате метод \verb|connect| возвращает объект специального класса (database handler),
через который будет происходить взаимодействие с базой:
\begin{minted}{perl}
$dbh = DBI->connect(
    $dsn, $user, $password,
    { RaiseError => 1, AutoCommit => 0 }
);
\end{minted}
В частности, метод \verb|do| позволяет делать SQL запросы:
\begin{minted}{perl}
$dbh->do($sql);
\end{minted}

В зависимости от того, с какой базой данных предстоит работать, расположение базы данных задаётся по-разному:
\begin{minted}{perl}
$dbh = DBI->connect(
    $data_source, $user, $password,
    {...}
);

# DBD::SQLite
$dbh = DBI->connect(
    "dbi:SQLite:dbname=dbfile", "", ""
);

# DBD::mysql
$dbh = DBI->connect(
    "DBI:mysql:database=$database;" . "host=$hostname;port=$port",
    $user,
    $password
);
\end{minted}
Формат, в котором необходимо представить данные для подключения к конкретной базе данных, подробно описывается в документации к модулю, который обеспечивает взаимодействие с ней.
\begin{minted}{http}
dbi:DriverName:database_name
dbi:DriverName:database_name@hostname:port
dbi:DriverName:database=DBNAME;host=HOSTNAME;port=PORT
\end{minted}

\subsection{Метод \texttt{do}}
Как уже было сказано, метод \verb|do| объекта database handler позволяет делать запросы к базе данных:
\begin{minted}{perl}
my $number_of_rows = $dbh->do(
    'DELETE FROM user WHERE age < 18
');
\end{minted}
В результате выполнения данного кода из базы данных будут удалены все пользователи, возраст которых меньше 18 лет.

Удалить всех пользователей, которые имеют определённое имя, можно с помощью следующего кода:
\begin{minted}{perl}
my $name = <>;
$dbh->do(
    "DELETE FROM user WHERE name = '$name'"
);
\end{minted}
Этот код содержит уязвимость, известную как SQL injection.

\subsection{SQL injection}
SQL injection заключается в том, что специальным образом составленное значение переменной может привести к выполнению произвольного запроса к базе данных:
\begin{minted}{perl}
my $name = q{' OR (DELETE FROM log) AND '' = '};

$dbh->do(
    "DELETE FROM user WHERE name = '$name'"
);
\end{minted}

\begin{minted}{SQL}
DELETE FROM user WHERE name = ''
    OR (DELETE FROM log) AND '' = ''
\end{minted}

Чтобы безопасно производить запросы, переменную необходимо заэкранировать
с помощью метода \verb|quote|.
Экранированную переменную можно использовать в запросах, не опасаясь SQL инъекции:
\begin{minted}{perl}
  $name = $dbh->quote($name);
\end{minted}
Однако часто программисты забывают про это.

\subsection{Методы \texttt{prepare} и \texttt{execute}}
Более совершенный способ заключается в использовании методов \verb|prepare| и \verb|execute|:
\begin{minted}{perl}
my $sth = $dbh->prepare(
    'DELETE FROM user WHERE name = ?'
);
\end{minted}
Метод \verb|prepare| готовит запрос к исполнению. Поскольку символ ? не является валидным в SQL, он используется как метка параметров. При выполнении метода \verb|execute| все такие метки заменяются на экранированные значения аргументов:
\begin{minted}{perl}
    $sth->execute('Vadim');
\end{minted}
Следует отметить, что в некоторых базах данных метод \verb|prepare| реализован на стороне базы данных,
а значит при вызове этого метода происходит обращение к базе.
Это позволяет увеличить производительность в случае большого числа однотипных запросов.

\subsection{Методы \texttt{fetchrow} и \texttt{fetchall}}
После того, как метод \verb|execute| выполнен, необходимо как-то извлечь выбранные в запросе данные.
Методы \verb|fetchrow_array|, \verb|fetchrow_arrayref| и \verb|fetchrow_hashref| возвращают данные по одной записи.
\begin{minted}{perl}
my @ary      = $sth->fetchrow_array();
my $ary_ref  = $sth->fetchrow_arrayref();
my $hash_ref = $sth->fetchrow_hashref();

while (@row = $sth->fetchrow_array()) {
    print "@row\n";
}
\end{minted}
Разница между этими методами следующая: \verb|fetchrow_array| возвращает запись как массив,
\verb|fetchrow_arrayref|~--- как ссылку на массив, \verb|fetchrow_hashref|~--- как ссылку на хеш.

Получить все данные результата запроса можно с помощью методов \verb|fetchall_arrayref| и \verb|fetchall_hashref|:
\begin{minted}{perl}
my $ary = $sth->fetchall_arrayref;
# [ [...], [...], [...] ]

my $ary = $sth->fetchall_arrayref({});
# [ {...}, {...}, {...} ]
\end{minted}
Дополнительный параметр позволяет указать, в каком виде должны быть представлены записи.

Например, если передать в качестве этого параметра любой хеш, \verb|fetchall_arrayref| вернет массив хешей, вызывая внутри себя метод \verb|fetchrow_hashref|:
\begin{minted}{perl}
$tbl_ary_ref = $sth->fetchall_arrayref({
    foo => 1,
    BAR => 1,
});
\end{minted}
В дополнительном параметре можно указать номера колонок или их имена. Чтобы указать, колонки с какими номерами необходимо вернуть, в качестве дополнительного параметра нужно передать массив с этими номерами. Отрицательные номера значат номер колонки с конца.
\begin{minted}{perl}
  $tbl_ary_ref = $sth->fetchall_arrayref(
      [0]
  );

  $tbl_ary_ref = $sth->fetchall_arrayref(
      [-2, -1]
  );
\end{minted}

Ещё раз следует отметить, что именно тип параметра задаёт тип возвращаемых данных, а не сам параметр. В случае, когда нужно указать конкретные колонки и вернуть результат в виде массива хешей, в качестве дополнительного параметра необходимо передать хеш, значения в котором произвольны, а ключи~--- это в точности имена требуемых колонок:
\begin{minted}{perl}
$sth->fetchall_hashref('id');
# { 1 => {...}, 2 => {...} }
\end{minted}

Метод \verb|fetchall_hashref| всегда возвращает хеш, значения в котором тоже являются хешами. Ключом для некоторой строки будет являться id:
\begin{minted}{perl}
$sth->fetchall_hashref([ qw(foo bar) ]);

{
    1 => { a => {...}, b => {...} },
    2 => { a => {...}, b => {...} },
}
\end{minted}

\subsection{Методы \texttt{selectrow} и \texttt{selectall}}
Методы \verb|selectrow_array|, \verb|selectrow_arrayref|,  \verb|selectrow_hashref| позволяют сразу сделать запрос и вернуть одну строчку:
\begin{minted}{perl}
$dbh->selectrow_array(
    $statement, \%attr, @bind_values
);

$dbh->selectrow_arrayref(
    $statement, \%attr, @bind_values
);

$dbh->selectrow_hashref(
    $statement, \%attr, @bind_values
);
\end{minted}

Если ожидается много строчек в результате запроса, необходимо использовать один из методов
\verb|selectall|: \verb|selectall_array|, \verb|selectall_arrayref| или \verb|selectall_hashref|:
\begin{minted}{perl}
$dbh->selectall_arrayref(
    $statement, \%attr, @bind_values
);

$dbh->selectall_hashref(
    $statement, $key_field, \%attr, @bind_values
);

$dbh->selectall_arrayref(
    "SELECT ename FROM emp ORDER BY ename",
    { Slice => {} }
);
\end{minted}

\subsection{Обработка ошибок}
% TODO 41-00 Лектор говорит, что не уверен в своих словах про RaiseError. Этот фрагмент нужно набрать по документации.

\begin{minted}{perl}
$dbh = DBI->connect(
    "dbi:DriverName:db_name", $user, $password,
    { RaiseError => 1 }
);
\end{minted}
Когда ошибка происходит при исполнении запроса, данные об ошибке доступны через database handler. Код ошибки и текст ошибки можно узнать используя методы \verb|err| и \verb|errstr|:
\begin{minted}{perl}
$dbh->err;
$dbh->errstr;
\end{minted}
Конкретные коды ошибки зависит от используемой базы данных.

\subsection{Транзакции}
Транзакция~--- группа запросов, для которых гарантируется их атомарное исполнение,
то есть либо каждый запрос из группы будет выполнен, либо не будет выполнен ни один.
\begin{minted}{perl}
$dbh = DBI->connect(
    "dbi:DriverName:db_name", $user, $password,
    { AutoCommit => 1 }
);

$dbh->begin_work;
$dbh->rollback;
$dbh->commit;
\end{minted}

Классический пример транзакции~--- перевод денег со счета на счёт.
В этом случае операции <<снять деньги с первого счета>> и <<положить деньги на второй счёт>> должны быть или обе выполнены, или обе не выполнены.

% TODO про метод connect и AutoCommit (43-30)

Метод \verb|last_insert_id| возвращает id последней созданной строки в текущей транзакции:
\begin{minted}{perl}
  $dbh->do('INSERT INTO user VALUES(...)');

  my $user_id = $dbh->last_insert_id(
      $catalog, $schema, $table, $field, \%attr
  );
\end{minted}

\section{Object-relational mapping}
Существует ряд подходов, которые заключаются в использовании классов, отвечающих за взаимодействие с базой. Это так называемый ORM (Object-relational mapping), который позволяет работать с данными в базе абстрагировано.

\subsection{Модуль DBIx::Class}
Один из таких слоёв абстракции обеспечивает модуль DBIx::Class. Для того, чтобы работать с базой данных через DBIx::Class, ему необходимо <<объяснить>>, какие таблицы есть в базе и как они связаны между собой. После этого работа с данными будет происходить с использованием объектно-ориентированного подхода.
\begin{minted}{perl}
  package Local::Schema::User;
  use base qw(DBIx::Class::Core);

  __PACKAGE__->table('user');
  __PACKAGE__->add_columns(
      id => {
          data_type => 'integer',
          is_auto_increment => 1,
      },
      name => {
          data_type => 'varchar',
          size      => '100',
      },
      superuser => {
          data_type => 'bool',
      },
  );
\end{minted}

\subsection{Метод \texttt{resultset}}
Основные объекты, которыми манипулирует DBIx::Class, это resultset'ы.
Их следует воспринимать как потенциальный массив строк и как запрос. Например:
\begin{minted}{perl}
  my $resultset = $schema->resultset('User');
  my $resultset2 = $resultset->search({age => 25});
\end{minted}

До тех пор, пока не будет выполнен метод \verb|next|,
реальных запросов к базе данных не производится.
До этого момента запрос только формируется, подготавливается к исполнению на базе.
Метод \verb|next| используется, чтобы получать данные от базы и выводить их построчно:
\begin{minted}{perl}
while (my $user = $resultset->next) {
    print $user->name . "\n";
}
\end{minted}

Сразу все строки можно вернуть с помощью метода \verb|all|:
\begin{minted}{perl}
  print join "\n", $resultset2->all();
\end{minted}

\subsection{Метод \texttt{search}}
После использования метода \texttt{resultset} значение, которое он возвратил, соответствует запросу на получение всей таблицы. Если необходимо указать условия на требуемые строки, их можно задать используя метод \texttt{search}:
\begin{minted}{perl}
$rs = $rs->search({
    age => {'>=' => 18},
    parent_id => undef,
});
\end{minted}
В качестве параметра метод \texttt{search} принимает хеш с условиями на значения полей. Если в качестве значения к некоторому ключу также будет передан хеш, то его ключ будет воспринят как оператор, а значение~--- как операнд. Например:
\begin{minted}{perl}
  @results = $rs->all();
  @results = $rs->search(...);
  $rs = $rs->search(...);
  $rs = $rs->search_rs(...);
\end{minted}
В качестве второго параметра можно задать сортировку результатов запроса и так далее:
\begin{minted}{perl}
$rs = $rs->search(
    { page => { '>=' => 18 } },
    { order_by => { -desc => [qw(a b c)] } },
);
\end{minted}
Если условия на значения полей не требуются, в качестве первого параметра следует отправить \verb|undef| или пустой хеш. С помощью второго параметра можно ограничить число строк в результате следующим образом:
\begin{minted}{perl}
  $rs = $rs->search(undef, { rows => 100 });
\end{minted}
Чтобы задать два условия на одно поле (хеш не может содержать два элемента с одинаковыми ключами), нужно использовать альтернативный синтаксис~--- каждую пару ключ-значение поместить в свой хеш и передать массив получившихся хешей:
\begin{minted}{perl}
# :-(
$rs = $rs->search({
    age => {'>=' => 18},
    age => {'<' => 60},
});

# :-)
$rs = $rs->search([
    { age => {'>=' => 18} },
    { age => {'<'  => 60} },
]);
\end{minted}

\subsection{Методы \texttt{find}, \texttt{single}}
Метод \verb|find| позволяет сразу вернуть не массив, а единственную строчку. Он поддерживает два синтаксиса: передать хеш, как в случае \verb|search|, или непосредственно передать значение первичного ключа:
\begin{minted}{perl}
  my $rs = $schema->resultset('User');

  $user = $rs->find({ id => 81858 });
  $user = $rs->find(81858);
\end{minted}
Фактически \verb|find| внутри себя делает следующее: вызывает метод \verb|search| и сразу вызывает метод \verb|single|:
\begin{minted}{perl}
  $user = $rs->search({id => 81858})->single();
\end{minted}
Метод \verb|single|, если результат представляет из себя одну строку, возвращает ее, и, если результат пустой, возвращает \verb|undef|. В случае, если результат содержит несколько строк, будет брошено предупреждение и возвращена первая строка.

\subsection{Метод \texttt{count}}
Метод \verb|count| возвращает количество строк в результате:
\begin{minted}{perl}
my $count = $schema->resultset('User')->search({
    name => 'name',
    age => 18,
})->count();
\end{minted}
Преимущество этого метода состоит в том, что не нужно выделять память для строк в результате, когда нужно знать только их количество.


\subsection{Метод \texttt{search}: продолжение}
Если при вызове метода \verb|search| в хеше с условиями в качестве значения подать не строку, а ссылку на строку, SQL код в этой строке будет вызван как есть, а не заэкранирован:
\begin{minted}{perl}
$resultset->search({
    date => { '>' => \'NOW()' },
});
\end{minted}
Если в качестве первого параметра метода \verb|search| передать ссылку на массив, содержащий SQL запрос с метками параметров (вопросительные знаки) и значениями, которые необходимо поставить на место этих меток, этот запрос будет исполнен, причём все значения будут автоматически экранированы:
\begin{minted}{perl}
$rs->search(
    \[ 'YEAR(date_of_birth) = ?', 1979 ]
);
\end{minted}
Также можно использовать \verb|or| и \verb|and| следующим образом:
\begin{minted}{perl}
my @albums = $schema->resultset('Album')->search({
    -or => [
        -and => [
            artist => { 'like', '%Smashing Pumpkins%' },
            title  => 'Siamese Dream',
        ],
        artist => 'Starchildren',
    ],
});
\end{minted}

\subsection{Связи между таблицами}
Следующий пример демонстрирует как может быть задана связь один ко многим:
\begin{minted}{perl}
  package Local::Schema::User;
  use base qw(DBIx::Class::Core);

  __PACKAGE__->table('user');
  __PACKAGE__->has_many(
      dogs => 'Local::Schema::Dog',
      'user_id'
  );

  package Local::Schema::Dog;
  use base qw(DBIx::Class::Core);

  __PACKAGE__->table('dog');
  __PACKAGE__->belongs_to(
      user => 'Local::Schema::User',
      'user_id'
  );
\end{minted}
Это позволяет, например, запрашивать всех собак, принадлежащих определённому пользователю с помощью вызова одного метода:
\begin{minted}{perl}
  $user = $schema->resultset('User')->find(81858);

  foreach my $dog ($user->dogs) {
    print join(' ', $dog->id, $dog->user->id);
  }
\end{minted}
Связи также можно использовать в запросах. В этом случае во втором параметре метода \verb|search| необходимо указать используемую связь:
\begin{minted}{perl}
  $rs = $schema->resultset('Dog')->search({
    'me.name' => 'Sharik',
    'user.name' => 'Vadim',
  }, {
    join => 'user',
  });
\end{minted}
Хотя \verb|join| и объединяет несколько таблиц, в полученных объектах содержатся данные лишь из одной таблицы, а чтобы получить данные из другой таблицы будет делаться дополнительный запрос. Получить все данные из связанных таблиц одним запросом можно, если вместо \verb|join| использовать \verb|prefetch|:
\begin{minted}{perl}
foreach my $user ($schema->resultset('User')) {
    foreach my $dog ($user->dogs) {
        # ...
    }
}

$rs = $schema->resultset('User')->search({}, {
    prefetch => 'dogs', # implies join
});
\end{minted}

\subsection{Custom result and resultset methods}
В DBIx::Class можно в ResultSet-классы добавлять дополнительные методы для работы с наборами строк, в частности, поиск по заданным критериям и сортировку:
\begin{minted}{perl}
  my @women = $schema->resultset('User')->
      search_women()->all();
\end{minted}
Также можно добавить дополнительные методы для Result-классов:
\begin{minted}{perl}
package Local::Schema::ResultSet::User;

sub search_women {
    my ($self) = @_;

    return $self->search({
        gender => 'f',
    });
}
\end{minted}
Это позволяет инкапсулировать часть логики внутри данных классов.

% TODO 1:15:00 пропущено
\subsection{Методы \texttt{new\_result}, \texttt{create}}
Метод \verb|new_result| ResultSet-класса возвращает новый объект. Этот объект не будет добавлен в базу данных до тех пор, пока не будет выполнен его метод \verb|insert|. Например:
\begin{minted}{perl}
my $user = $schema->resultset('User')->new_result({
    name => 'Vadim',
    superuser => 1,
});

$user->insert();
\end{minted}

Метод \verb|create| ResultSet-класса позволяет за один вызов создать объект и поместить его в базу:
\begin{minted}{perl}
my $artist = $artist_rs->create({
    artistid => 4, name => 'Blah-blah', cds => [
        { title => 'My First CD', year => 2006 },
        { title => 'e.t.c', year => 2007 },
    ],
});
\end{minted}
Интересно, что с помощью \verb|create| можно сразу указать и связи. В этом случае необходимые строки в других таблицах базы данных будут созданы автоматически.

\subsection{Методы \texttt{update} и \texttt{delete}}
Метод \verb|update| можно использовать после внесения изменения, чтобы закрепить его в базе данных:
\begin{minted}{perl}
  $result->last_modified(\'NOW()')->update();
\end{minted}
Также можно сразу передать хеш с данными, которые нужно изменить:
\begin{minted}{perl}
  $result->update({ last_modified => \'NOW()' });
\end{minted}
Метод \verb|delete| позволяет удалить объект из базы данных.
\begin{minted}{perl}
  $user->delete();
\end{minted}

\subsection{Связь многие ко многим}
% Диаграмма
% TODO пропущено

Часто таблица, которая используется для реализации связи многие-ко-многим, состоит всего из двух столбцов и сама по себе не интересна. Модуль DBIx::Class позволяет удобно работать с подобного рода связями:
\begin{minted}{perl}
\end{minted}

\subsection{Отладочный режим}
Включить отладочный режим можно с помощью следующего метода:
\begin{minted}{perl}
  $schema->storage->debug(1);
\end{minted}
В этом случае результаты всех запросов будут выведены в \verb|STDOUT|.

DBIx::Class внутри себя использует модуль DBI, о котором была речь в начале лекции. Получить непосредственно доступ к database handler можно следующим образом:
\begin{minted}{perl}
    $schema->storage->dbh();
\end{minted}

\subsection{Генерирование схемы на основе готовой базы} % 1:27:30
Класс DBIx::Class::Schema::Loader позволяет с помощью функции \verb|make_schema_at| сгенерировать схему на основе готовой базы:
\begin{minted}{perl}
use DBIx::Class::Schema::Loader qw(
    make_schema_at
);

make_schema_at(
    'My::Schema',
    {
        debug => 1,
        dump_directory => './lib',
    },
    [ 'dbi:Pg:dbname="foo"', 'user', 'pw' ]
);
\end{minted}
Этот класс подписывает сгенерированные файлы контрольной суммой, чтобы в следующий раз он мог сверху накатить изменения и не сломать добавленный вручную код. Если схема была изменена вручную, такое автоматическое обновление становится недоступным.

Ручное редактирование схемы имеет следующие преимущества:
\begin{itemize}
  \item Колонкам можно давать произвольные имена.
  \item Схеме можно не <<рассказывать>> о некоторых колонках (которые база данных обслуживает самостоятельно).
  \item Можно указывать разные настройки для разных таблиц.
\end{itemize}
Для того, чтобы перейти в режим ручного редактирования, достаточно просто удалить контрольные суммы из файлов.

Кроме модуля существует утилита dbicdump, которая позволяет делать то же самое без написания perl-скрипта:
\begin{minted}{sh}
  dbicdump -o dump_directory=./lib \
           -o debug=1 \
           My::Schema \
           'dbi:Pg:dbname=foo' \
           myuser \
           mypassword
\end{minted}

\subsection{SQL::Translator}
Возможен другой подход~--- создать схему, а потом создать все необходимые таблицы и связи в базе данных с помощью следующей команды:
\begin{minted}{perl}
  $schema->deploy();
\end{minted}
При этом следует учитывать, что не все возможности базы данных могут быть задействованы при таком способе. % TODO про автотест домашки

\subsection{Изменение схемы в режиме реального времени} % на основе ответа на вопрос
Изменение схемы в ходе работы сервиса представляет из себя серьёзную проблемы. Во-первых, для достаточно больших таблиц создание новой колонки с некоторым значением по умолчанию занимает много времени. Чтобы внести такие изменения, не останавливая работу сервиса, изменение схемы происходит в несколько этапов:
\begin{enumerate}[nosep]
  \item Добавить колонку с пустым значением по умолчанию.
  \item Код переписывается таким образом, чтобы он мог работать и со старой схемой, и с новой.
  \item Постепенно вносить данные и заполнять стандартное значение.
\end{enumerate}
Иногда бывает разумнее (если сервис небольшого размера) приостановить работу сервиса, внести изменения и запустить снова. Но для сколько-нибудь крупных проектов это неприемлемо.

\section{Memcached}
Memcached~--- примитивное хранилище пар ключ-значение. Memcached в этом смысле не совсем является базой данных. В Memcached доступны несколько команд~--- для доступа к данным и добавления данных. Для каждой пары ключ-значение можно задать время хранение, что позволяет использовать Memcached в качестве кэша (откуда и название).

Модуль Cache::Memcached::Fast позволяет хранить данные на нескольких серверах (\verb|weight| задает вес этого сервера, данные располагаются на серверах пропорционально значениям весов). Например:
\begin{minted}{perl}
use Cache::Memcached::Fast;

my $memd = Cache::Memcached::Fast->new({
    servers => [
        { address => 'localhost:11211', weight => 2.5 },
        '192.168.254.2:11211',
        '/path/to/unix.sock'
    ],
    namespace => 'my:',
    connect_timeout => 0.2,
    # ...
});
\end{minted}
После этого можно использовать 4 основных операции:
\begin{minted}{perl}
  $memd->add('skey', 'text');
  $memd->set('nkey', 5, 60);
  $memd->incr('nkey');
  $memd->get('skey');
\end{minted}
При этом операция инкремента обеспечивает атомарность. Атомарность обеспечивается следующим образом. При чтении данных возвращается значение и специальный ключ. Этот ключ можно использовать, чтобы выполнить \verb|set|, но только в том случае, если ключ не изменился (ключ меняется при изменении данных). Если ключ изменился, данные нужно будет считать заново, обработать и вновь попытаться сохранить. Это будет продолжаться до тех пор, пока имеет место такая <<гонка>>.

%\setcounter{chapter}{7}
\chapter{Веб-программирование}
\section{Протокол HTTP}
Простейший HTTP-запрос состоит из метода, URL-адреса, версии протокола и хоста, на который происходит обращение:
\begin{minted}[frame=leftline,framesep=1em,linenos,firstline=1]{http}
GET / HTTP/1.1
Host: search.cpan.org
\end{minted}
Ответ выглядит примерно так же:
\begin{minted}[frame=leftline,framesep=1em,linenos,firstline=1]{http}
HTTP/1.1 200 OK
Date: Mon, 13 Apr 2015 20:19:35 GMT
Server: Plack/Starman (Perl)
Content-Length: 3623
Content-Type: text/html

__CONTENT__
\end{minted}
Сервер возвращает протокол, по которому он согласен работать. Дальнейшие запросы должны быть произведены с использованием этой версии протокола. После версии протокола идет код возврата, о котором речь пойдет позже.
Так выглядят все возможные запросы и ответы по HTTP.

Формат ответа следующий:
\begin{verbatim}
 Response      = Status-Line  ; Section 6.1
 *(( general-header           ; Section 4.5
 | response-header            ; Section 6.2
 | entity-header ) CRLF)      ; Section 7.1
 CRLF
 [ message-body ]             ; Section 7.2

 Status-Line = HTTP-Version SP Status-Code SP
               Reason-Phrase CRLF
\end{verbatim}
Он состоит из \verb|Status-Line| (первая строчка), затем идут обязательные заголовки \verb|general-header| (в версии протокола 1.1 единственный обязательный заголовок это \verb|Host|), следом идут \verb|response-header| или \verb|request-header|.
Дополнительные заголовки содержатся в \verb|response-header| и могут быть какими угодно, например:
информация о том, что страница взята из кеша или сгенерирована на лету, время генерации страницы и тому подобное.
Часто в платных API в этих заголовках передается информация о количестве доступных для использования запросов.

Status-Code бывают следующие:
\begin{itemize}
\item \verb|1**|~--- информационные сообщения от сервера клиенту;
\item \verb|2**|~--- успешная обработка запроса;
\item \verb|3**|~--- контент находится в другом месте или не изменялся:
    \begin{itemize}
    \item \verb|301|~--- контент перемещен навсегда;
    \item \verb|302|~--- контент перемещен временно;
    \end{itemize}
\item \verb|4**|~--- ошибка обработки запроса (клиент неправильно сформировал пакет):
    \begin{itemize}
    \item \verb|403|~--- документ есть, но для доступа нужна авторизация;
    \item \verb|404|~--- документа нет;
    \end{itemize}
\item \verb|5**|~--- ошибка обработки запроса (проблема на сервере):
    \begin{itemize}
    \item \verb|504|~--- Gateway Time Out;
    \item \verb|502|~--- сервер не может принять запрос;
    \end{itemize}
\end{itemize}
В версии протокола 1.1 \verb|HTTP-message Request| и \verb|HTTP-message Response| были унифицированы:
\begin{verbatim}
 HTTP-message   = Request | Response

 generic-message = start-line
 *(message-header CRLF)
 CRLF
 [ message-body ]

 start-line      = Request-Line | Status-Line
\end{verbatim}

\begin{verbatim}
message-header = field-name ":" [ field-value ]
field-name     = token
field-value    = *( field-content | LWS )
field-content  = <the octets making up the field-
value and consisting of either *text or combinations
of token, separators, and quoted-string>
\end{verbatim}
То есть при передаче JPG-файла на сервер его заголовки будут практически такими же, как и заголовки запроса на скачку этого же файла с сервера. Это сильно упрощает структуру приложений, так как можно использовать один парсер и для Request, и для Response.

HTTP изначально и на данный момент является однонаправленным протоколом, то есть запросы может формировать только клиент. Если на сервере что-то случилось, клиент об этом не будет знать до тех пор, пока не отправит запрос. На самом деле это довольно большая проблема: постоянно нужно опрашивать сервер, а не случилось ли там что-то и так далее.
Чтобы решить эту проблему был введен \emph{long polling}: если для клиента, пославшего запрос, есть сообщение, то сервер отвечает сразу; а если нет, то отвечает тогда, когда это сообщение появляется, но не позднее, чем через минуту. Если сообщение не пришло в течение минуты, сервер отправляет сообщение, что ничего нет.

В протоколе HTTP существуют следующие методы.
\begin{itemize}
  \item \textbf{GET} позволяет получить информацию от сервера, тело запроса всегда остается пустым.
  \item \textbf{HEAD}: аналогичен GET, но тело ответа остается всегда пустым, позволяет проверить доступность запрашиваемого ресурса и прочитать HTTP-заголовки ответа.
  \item \textbf{POST}: позволяет загрузить информацию на сервер, по смыслу изменяет ресурс на сервере, но зачастую используется и для создания ресурса на сервере, тело запроса содержит изменяемый/создаваемый ресурс.
  \item \textbf{PUT}: аналогичен POST, но по смыслу занимается созданием ресурса, а не его изменением, тело запроса содержит создаваемый ресурс.
  \item \textbf{DELETE}: удаляет ресурс с сервера.
  \item И некоторые другие.
\end{itemize}
Методы просто передаются приложению и различия между методами должны быть реализованы на уровне приложения. Методы необходимы для разделения запросов по смыслу.

HTTP-протокол может быть использован как транспортный протокол:
\begin{itemize}
  \item \textbf{XML-RPC} (RPC~--- удаленный вызов процедур) представляет собой протокол, основанный на XML поверх HTTP, и позволяет обращаться к библиотеке по сети. В XML-запросе содержится имя процедуры, данные для авторизации и параметры, которые нужно передать процедуре. Результат отправляется клиенту также в виде XML.

  \item \textbf{SOAP} представляет собой небольшое расширение \textbf{XML-RPC}. В WSDL-файлах описываются все возможные процедуры со всеми возможными параметрами, с которыми можно их вызвать. Соответственно, клиент должен скачать WSDL файл и проверить валидность своего запроса перед отправкой. Часто SOAP используется при построении web-архитектуры: каждый сервер характеризуется своим WSDL файлом, то есть набором доступных функций. После этого задача управления архитектурой решается с помощью языка BPEL (Business Process Execution Language).
Фактически вся бизнес-логика настраивается с помощью BPEL самими менеджерами. Существенный минус такого подхода~--- крайне высокая нагрузка на центральный процессор из-за валидации.

  \item \textbf{WebSocket} представляет собой надстройку над HTTP, которая позволяет поддерживать соединение между клиентом и сервером. Если соединение было установлено, оно не обрывается до тех пор, пока не произойдет сбоя оборудования.
Web-сокеты работают только в современных браузерах.
\end{itemize}

Протокол HTTP может быть перехвачен и прочтен без каких-либо проблем, что представляет серьезную угрозу безопасности. Для решения этой проблемы был внедрен HTTPS~--- протокол, построенный поверх HTTP с использованием SSL.
То есть до начала обмена данными клиент отправляет запрос на установку защищённого соединения. В ответ сервер отправляет сертификат и случайную цифру, на основе которых генерируется цифровая подпись на стороне сервера и на стороне клиента. С этого момента все данные шифруются, используя эту цифровую подпись. При этом каждый клиент, начиная новую сессию, получает свой уникальный ключ для шифрования. Аналогично был создан WSS, защищенный вариант WebSocket. Использование HTTPS является минимумом, который нужно обеспечивать с точки зрения безопасности, если на сервисе есть авторизация.

\section{Взаимодействие сервера и приложения} %11 33:51
Изначально сервера могли отдавать только статические страницы и не могли создавать страницы динамически. Когда мощности компьютеров увеличились, был создан интерфейс CGI (Common Gateway Interface), который представляет собой интерфейс между сервером и приложением. Работало это следующим образом: сервер принимает запрос, если запрос идет на динамически генерируемую страницу, параметры запроса перенаправляются в приложение, приложение их обрабатывает и генерирует страницу, которая затем возвращается клиенту.

Эта технология была очень востребованной в момент начала зарождения динамических страниц, но имела существенный недостаток: приложение запускалось с нуля для каждого запроса, а не висело в памяти. Это создавало высокую нагрузку на сервер в случае большого количества запросов.

В результате появилась надстройка \verb|mod_perl| для Apache, которая запускала приложение на Perl внутри себя и оставляла его в памяти на все время работы сервера.
Веб-сервер сам понимает сколько необходимо запустить приложений внутри себя, чтобы распараллелить обработку всех запросов.
Таким образом, не нужно было тратить время на запуск приложения для каждого запроса и данное решение работало гораздо быстрее, чем CGI.

\section{Безопасность в приложениях} %45 1:38:20
Основные уязвимости:
\begin{itemize}
  \item \textbf{Cross-site scripting (XSS)}: злоумышленник отправляет на сервер вредоносный скрипт (например, набирает скрипт в поле для имени в социальной сети), который потом будет исполнен на компьютере пользователей.
Чтобы избежать этой уязвимости в приложении, все спецсимволы HTML во входных данных должны экранироваться при отображении пользователям.

  \item \textbf{Межсайтовая фальсификация запросов (CSRF)}: жертва заходит на сайт злоумышленника, при загрузке которого (например, указан в качестве URL картинки) от ее имени тайно делается запрос на другой сервер (например, сервер банка), где жертва уже авторизована, и выполняется вредоносная операция (например, деньги жертвы переводятся на счет злоумышленника).

  Чтобы защитить приложение от такой уязвимости, все модифицирующие операции должны исполняться в ответ на POST (так как браузер загружает картинки с помощью GET). Также каждую форму нужно защищать специальными токенами: каждая страница с формой должна иметь также набор секретных ключей, которые вместе с формой передаются на сервер. Такая защита основана на том, что злоумышленник не сможет сделать два запроса и анализировать ответ одного из них (браузеры не позволяют обмениваться данными сайтам из разных доменов).

  \item \textbf{Response-splitting}: на уязвимом сервере существует возможность вставлять произвольные пользовательские данные в заголовки HTML ответов. В результате, можно поместить туда такие данные, что компьютер жертвы воспримет HTML ответ как два HTML ответа, причем содержание контролируется злоумышленником.

  \item \textbf{Innumeration}: перебор id в запросах с целью получения личных данных.
  \item \textbf{Sql-injection}~--- внедрение в запрос произвольного SQL-кода.
  \item \textbf{Remote Code Execution}~--- исполнение произвольного кода. Например, если данные клиента используются в вызовах \verb|eval|, \verb|open| и \verb|system|, то существует риск, что на сервере можно будет дистанционно исполнить произвольный код.
\end{itemize}

%\include{lectures/L9/L9}
%\include{lectures/L10/L10}
%\include{lectures/L11/L11}
%\include{lectures/L12/L12}
\end{document}
